\renewcommand{\abstractname}{Abstract} % Veränderter Name für das Abstract
\chapter*{Abstract}
\begin{addmargin}[1.5cm]{1.5cm}        % Erhöhte Ränder, für Abstract Look
\thispagestyle{plain}                  % Seitenzahl auf der Abstract Seite

\begin{center}
\small\textit{- Deutsch -}             % Angabe der Sprache für das Abstract
\end{center}

\vspace{0.25cm}

Das Landeszentrum für Datenverarbeitung betreibt, neben anderen IT-Dienst\-leis\-tun\-gen, ein Schwachstellen- und Bedrohungsmanagement. Für die Bereitstellung der Dienstleistungen betreibt das Landeszentrum für Datenverabeitung eine umfangreiche Infrastruktur, weshalb das Schwachstellenmanagement mithilfe einer Anwendung entlastet werden soll.\\
In der Arbeit werden zunächst Grundlagen der Informationssicherheit und Anwendungsentwicklung beschrieben. In Eigenrecherche werden potenzielle Metriken erarbeitet, die genutzt werden können, um das Schwachstellenmanagement zu entlasten. Die Metriken werden den betroffenen Fachbereichen präsentiert und im Anschluss werden Anforderungen an die Anwendung durch eine Anforderungsanalyse ermittelt. Die gestellten Anforderungen fließen in das Konzept ein, das zur Erstellung eines Prototyps verwendet wird. Der Prototyp wird in Python implementiert und verwendet unter anderen Nmap zur Informationsbeschaffung und MariaDB als relationales Datenbank Managementsystem. In der Evaluation werden die Ergebnisse des Prototyps in einer Testumgebung aufgezeigt und untersucht.\\
Die durchgeführten Tests zeigen die Funktionalität des Prototyps und den Nutzen der Anwendung. Die entwickelte Anwendung sorgt für eine Reduktion des Aufwands, den das Schwachstellenmanagement leisten muss und verbessert die Effektivität. Neben den ermittelten Metriken, die zur Verbesserung der Sicherheit der Infrastruktur genutzt werden können, liefert die Anwendung Möglichkeiten für das Berichtswesen und Analyse.

\end{addmargin}

\clearpage

\begin{addmargin}[1.5cm]{1.5cm}        % Erhöhte Ränder, für Abstract Look
\thispagestyle{plain}   
\begin{center}
\small\textit{- English -}             % Angabe der Sprache für das Abstract
\end{center}

\vspace{0.25cm}

In addition to other information technology services, the Landeszentrum für Datenverarbeitung operates a vulnerability and threat management. To provide these services, the Landeszentrum für Datenverarbeitung operates an extensive infrastructure, which is why vulnerability management is to be relieved with the help of an application.\\
The paper first describes the basics of information security and application development. In self-research, potential metrics were developed that can be used to relieve the vulnerability management. The metrics were presented to the concerned departments and requirements for the application were then identified and analyzed through an requirements analysis. The requirements provided are incorporated into the concept, which is used to create a prototype. The prototype is implemented in Python and uses, among others, Nmap for information retrieval and MariaDB as relational database management system. In the evaluation, the results of the prototype were shown and examined in a test environment.
The tests performed show the functionality of the prototype and the usefulness of the application. The developed application provides reduction of the effort the vulnerability management has to accomplish and improves the effectiveness. In addition to the identified metrics that can be used to improve the security of the infrastructure, the application provides reporting and analysis capabilities.

\end{addmargin}