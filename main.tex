%	ACHTUNG: Für das Erstellen des Literaturverzeichnisses wird das modernere Paket biblatex
%			 in Kombination mit biber verwendet -- nicht mehr das ältere BibTex!
% 			 Bitte stellen Sie ggf. Ihre TeX-Umgebung
% 			 entsprechend ein (z.B. TeXStudio: Einstellungen --> Erzeugen --> Standard Bibliographieprogramm: biber)
%
%bei TexMaker: Optionen -> TexMaker konfigurieren -> bei Bib(la)Tex muss biber.exe ausgewählt werden(befindet sich bei MikTex im Ordner)

\documentclass[
	12pt,
	top=25mm,
	bottom=20mm,
	left=35mm,
	right=25mm,
	DIV=12,
	headinclude=on,
	footinclude=off,
	parskip=half,
	bibliography=totoc,
	listof=entryprefix,
	toc=listof,
	pointlessnumbers,
	plainfootsepline
]{scrreprt}


	\oddsidemargin = 0pt %für Seitenrand links kleiner
	\marginparsep = 16pt %für Seitenrand rechts größer

%	Konfigurationsdatei einziehen
% !TEX root =  master.tex

%		LANGUAGE SETTINGS AND FONT ENCODING 
%
\usepackage[ngerman]{babel} 	% German language
\usepackage[utf8]{inputenc}
\usepackage[german=quotes]{csquotes} 	% correct quotes using \enquote{}
\usepackage[T1]{fontenc}

%\usepackage[english]{babel}   % For english language
%\usepackage{csquotes} 	% Richtiges Setzen der Anführungszeichen mit \enquote{}

% Zwei eigene Befehle zum Setzen von Autor und Titel. Ausserdem werden die PDF-Informationen richtig gesetzt.
\newcommand{\TitelDerArbeit}[1]{\def\DerTitelDerArbeit{#1}\hypersetup{pdftitle={#1}}}
\newcommand{\AutorDerArbeit}[1]{\def\DerAutorDerArbeit{#1}\hypersetup{pdfauthor={#1}}}
\newcommand{\Firma}[1]{\def\DerNameDerFirma{#1}}
\newcommand{\Kurs}[1]{\def\DieKursbezeichnung{#1}}

% Correct superscripts 
\usepackage{fnpct}

%		CALCULATIONS
%
\usepackage{calc} % Used for extra space below footsepline

%		BIBLIOGRAPHY SETTINGS
%
% Uncomment the next three lines for author-year-style with footnotes (Chicago)
%\usepackage[backend=biber, autocite=footnote, style=authortitle, dashed=false]{biblatex}%vorher authoryear
%Use Author-Year-Cites with footnotes
%\AdaptNoteOpt\footcite\multfootcite   %will add  separators if footcite is called multiple consecutive times 
%\AdaptNoteOpt\autocite\multautocite % will add  separators if autocite is called multiple consecutive times

% Uncomment the next line for IEEE-style 
% \usepackage[backend=biber, autocite=inline, style=ieee]{biblatex} 	% Use IEEE-Style (e.g. [1])

% Uncomment the next line for alphabetic style 
% \usepackage[backend=biber, autocite=inline, style=alphabetic]{biblatex} 	% Use alphabetic style (e.g. [TGK12])

% Uncomment the next two lines vor Harvard-Style 
\usepackage[natbib=true, backend=biber, style=authoryear, dashed=false]{biblatex} 	
%\DeclareLanguageMapping{german}{german-apa}


\DefineBibliographyStrings{ngerman}{  %Change u.a. to et al. (german only!)
	andothers = {{et\,al\adddot}},
}

\renewcommand{\bibleftparen}{[}
\renewcommand{\bibrightparen}{]}

\let\cite\parencite

%%%für die Umbrüche der URLs im Literaturverzeichnis
\apptocmd{\UrlBreaks}{\do\f\do\m}{}{}
\setcounter{biburllcpenalty}{9000}% Kleinbuchstaben
\setcounter{biburlucpenalty}{9000}% Großbuchstaben

\setlength{\bibparsep}{\parskip}	%add some space between biblatex entries in the bibliography
\addbibresource{bibliography.bib}	%Add file bibliography.bib as biblatex resource

%		FOOTNOTES 
%
% Count footnotes over chapters
\usepackage{chngcntr}
\counterwithout{footnote}{chapter}

\usepackage{tablefootnote} %damit Fußnoten in Tabellen richtig angezeigt werden

%	ACRONYMS
%%%
%%% WICHTIG: Installieren Sie das neueste Acronyms-Paket!!!
%%%
\makeatletter
\usepackage[printonlyused]{acronym}
\@ifpackagelater{acronym}{2015/03/20}
  {%
    \renewcommand*{\aclabelfont}[1]{\textbf{\textsf{\acsfont{#1}}}}
  }%
  {%
  }%
\makeatother

%		LISTINGS
\usepackage{listings}	%Format Listings properly
\usepackage{color} %für das Styling in anderen Farben
\usepackage{xcolor}

\renewcommand{\lstlistingname}{Quelltext}
\renewcommand{\lstlistlistingname}{Quelltextverzeichnis}

\lstset{numbers=left,
  numberstyle=\tiny,
  captionpos=t,
  basicstyle=\ttfamily\small,
  escapeinside={(*@}{@*)},
  literate= %für die Umlaute in Listings
  {Ö}{{\"O}}1
  {Ä}{{\"A}}1
  {Ü}{{\"U}}1
  {ß}{{\ss}}1
  {ö}{{\"o}}1
  {ü}{{\"u}}1
  {ä}{{\"a}}1
  {~}{{\textasciitilde}}1
}
\definecolor{codeblue}{rgb}{0.25,0.5,0.75}
\definecolor{codered}{rgb}{0.8,0,0}
\definecolor{codegreen}{rgb}{0,0.6,0} 
\lstdefinestyle{stylePython}{
    backgroundcolor=\color{white},       
    belowcaptionskip=1\baselineskip,
    breaklines=true,
    frame=L,
    xleftmargin=\parindent,
    language=Python,
    showstringspaces=false,
    basicstyle=\footnotesize\ttfamily,
    commentstyle=\color{codegreen},
    keywordstyle=\color{blue},
    identifierstyle=\color{black},
    stringstyle=\color{red},
    numberstyle=\tiny\color{blue},
    morekeywords={with, as}
}
\lstdefinestyle{styleLog}{
    backgroundcolor=\color{white},       
    belowcaptionskip=1\baselineskip,
    breaklines=true,
    frame=L,
    xleftmargin=\parindent,
    showstringspaces=false,
    basicstyle=\small
}
\lstdefinestyle{bashstyle}{
    language=bash,
    basicstyle=\ttfamily\footnotesize,
    keywordstyle=\color{blue}\bfseries,
    commentstyle=\color{gray}\itshape,
    stringstyle=\color{orange},
    backgroundcolor=\color{gray!10},
    frame=single,
    rulecolor=\color{black},
    breaklines=true,
    postbreak=\mbox{\textcolor{red}{$\hookrightarrow$}\space},
    showstringspaces=false,
    tabsize=2,
    captionpos=b,
}
\lstdefinestyle{styleSQL}{
    backgroundcolor=\color{white},
    belowcaptionskip=1\baselineskip,
    breaklines=true,
    frame=L,
    xleftmargin=\parindent,
    language=SQL,
    showstringspaces=false,
    basicstyle=\footnotesize\ttfamily,
    commentstyle=\color{codegreen},
    keywordstyle=\color{codeblue}\bfseries,
    identifierstyle=\color{black},
    stringstyle=\color{codered},
    numberstyle=\tiny\color{gray},
    morekeywords={IF, EXISTS, BEGIN, END, VARCHAR, INT, PRIMARY, KEY, FOREIGN, REFERENCES, CREATE, TABLE, INSERT, INTO, VALUES, SELECT, FROM, WHERE, JOIN, ON, GROUP, BY, ORDER, ASC, DESC, UPDATE, SET, DELETE},
}

\lstdefinelanguage{JavaScript}{
    morekeywords={typeof, new, true, false, catch, function, return, null, catch, switch, var, if, in, while, do, else, case, break},
    morecomment=[l]//,
    morecomment=[s]{/*}{*/},
    morestring=[b]",
    morestring=[b]',
}

\lstdefinelanguage{JSX}{
    language=JavaScript,
    morekeywords={
        className, render, return, props, state, useState, useEffect, useContext, useRef, useMemo, useCallback,
        React, Component, Fragment, setState, export, default, import, from, const, let, Grid, Box, SearchForm, DataGrid
    },
    sensitive=true,
    morecomment=[s]{/*}{*/},
    morestring=[b]',
    morestring=[b]"
    % Keine moredelim-Zeilen mehr
}
\lstdefinestyle{styleJavaSpring}{
    backgroundcolor=\color{white},       
    belowcaptionskip=1\baselineskip,
    breaklines=true,
    frame=L,
    xleftmargin=\parindent,
    language=Java,
    showstringspaces=false,
    basicstyle=\footnotesize\ttfamily,
    commentstyle=\color{codegreen},
    keywordstyle=\color{blue},
    identifierstyle=\color{black},
    stringstyle=\color{red},
    numberstyle=\tiny\color{blue},
    morekeywords={
        @Autowired, @Service, @Component, @Controller, @RestController, @Repository, @RequestMapping,
        @GetMapping, @PostMapping, @PutMapping, @DeleteMapping,
        SpringApplication, SpringBootApplication,
        ResponseEntity, RequestBody, PathVariable, RequestParam,
        List, Map, Optional
    }
}

\lstdefinestyle{styleReact}{
    backgroundcolor=\color{white},
    belowcaptionskip=1\baselineskip,
    breaklines=true,
    frame=L,
    xleftmargin=\parindent,
    language=JSX,
    showstringspaces=false,
    basicstyle=\footnotesize\ttfamily,
    commentstyle=\color{codegreen},
    keywordstyle=\color{blue},
    identifierstyle=\color{black},
    stringstyle=\color{red},
    numberstyle=\tiny\color{blue},
}

%		EXTRA PACKAGES
\usepackage{graphicx} % use various graphics formats
\usepackage[german]{varioref} 	% nicer references \vref
\usepackage{caption}	%better Captions
\usepackage{booktabs} %nicer Tabs
\usepackage{array}
%\newcolumntype{P}[1]{>{\raggedright\arraybackslash}p{#1}}
\usepackage{setspace} %für Zeilenabstand
\usepackage{hyphenat}
\usepackage{microtype}
\usepackage{amsmath}

% Tabelle
\usepackage{longtable}

%		ALGORITHMS
\usepackage{algorithm}
\usepackage{algpseudocode}
\renewcommand{\listalgorithmname}{Algorithmenverzeichnis }
\floatname{algorithm}{Algorithmus}

% Pages in landscape format
\usepackage{pdflscape}

%		FONT SELECTION: Entweder Latin Modern oder Times / Helvetica
%\usepackage{lmodern} %Latin modern font
\usepackage{mathptmx}  %Helvetica / Times New Roman fonts (2 lines)
\usepackage[scaled=.92]{helvet} %Helvetica / Times New Roman fonts (2 lines)

%PAGE HEADER / FOOTER
%Warning: There are some redefinitions throughout the master.tex-file!  DON'T CHANGE THESE REDEFINITIONS!
\RequirePackage[automark,headsepline,footsepline]{scrlayer-scrpage}
%\pagestyle{scrlayer-scrpage}
\renewcommand*{\pnumfont}{\upshape\sffamily}
\renewcommand*{\headfont}{\upshape\sffamily}
\renewcommand*{\footfont}{\upshape\sffamily}
\renewcommand{\chaptermarkformat}{}
\RedeclareSectionCommand[beforeskip=0pt]{chapter}
\clearscrheadfoot

\ifoot[\rule{0pt}{\ht\strutbox+\dp\strutbox}DHBW Karlsruhe]{\rule{0pt}{\ht\strutbox+\dp\strutbox}DHBW Karlsruhe}
\ofoot[\rule{0pt}{\ht\strutbox+\dp\strutbox}\pagemark]{\rule{0pt}{\ht\strutbox+\dp\strutbox}\pagemark}

\ohead{\headmark}

% 		HYPERREF
%
\usepackage[
	hidelinks=true % keine roten Markierungen bei Links
]{hyperref} %am Ende laden, ansonsten kommt es zu Fehlern

\makeatletter %damit Weiterleitung ins Glossar auf richtige Zeile erfolgt
 \newcommand{\linkdest}[1]{\Hy@raisedlink{\hypertarget{#1}{}}}
\makeatother

\setcounter{secnumdepth}{2}

%Behebung des Hook Errors beim Einbinden von lstlistings
\makeatletter %https://groups.google.com/g/comp.text.tex/c/nMUAQcY0N30
\ExplSyntaxOn
\cs_set_protected:Npn \para_end: {
\scan_stop:
\mode_if_horizontal:TF {
\mode_if_inner:F {
\tex_unskip:D
\hook_use:n{para/end}
\@kernel@after@para@end
\mode_if_horizontal:TF {
\if_int_compare:w 0 < \tex_lastnodetype:D
\tex_kern:D \c_zero_dim
\fi:
\tex_par:D
\hook_use:n{para/after}
\@kernel@after@para@after
}
{ \msg_error:nnnn { hooks }{ para-mode }{end}{horizontal} }
}
}
\tex_par:D
}
\cs_set_eq:NN \par \para_end:
\cs_set_eq:NN \@@par \para_end:
\cs_set_eq:NN \endgraf \para_end:
\ExplSyntaxOff
\makeatother

% adjust space before \paragraph{}
% https://tex.stackexchange.com/questions/4891/how-do-i-control-the-spacing-above-a-new-paragraph
\makeatletter
\renewcommand{\paragraph}{%
  \@startsection{paragraph}{4}%
  {\z@}{1.5ex \@plus 1ex \@minus .2ex}{-1em}%
  {\normalfont\normalsize\bfseries}%
}
\makeatother

% add "Quelltext" to the listings
% https://tex.stackexchange.com/questions/247038/change-lstlistoflistings-numbering
\makeatletter
\renewcommand{\l@lstlisting}[2]{%
  \@dottedtocline{1}{0em}{1.5em}{\lstlistingname\ #1}{#2}%
}
\makeatother

% add chapter to algorithm numbering
% https://tex.stackexchange.com/questions/124902/algorithm-with-chapter-number
\makeatletter 
\renewcommand\thealgorithm{\thechapter.\arabic{algorithm}} 
\@addtoreset{algorithm}{chapter} 
\makeatother



\begin{document}

%% BITTE GEBEN SIE HIER DEN TITEL UND DIE AUTORIN / DEN AUTOR DER ARBEIT AN!
%% DIESE INFORMATIONEN _MÜSSEN_ GESETZT SEIN, UM TITELBLATT, ABSTRACT UND
%% EIGENSTÄNDIGKEITSERKLÄRUNG AUTOMATISCH ANZUPASSEN!

\TitelDerArbeit{Entwicklung eines Testfallgenerators für verschiedene Eingangsformate in der Grundsteuer}
\AutorDerArbeit{Ante Babic}
\Firma{Oberfinanzdirektion Baden-Württemberg}
\Kurs{TINF24B2}
\newcommand{\Studiengang}{Informatik / Angewandte Informatik}
\newcommand{\AbgabeDatum}{29. September 2025}

\onehalfspacing %1,5-zeiliger Abstand zwischen allen Zeilen

%\begin{titlepage}
%	\begin{minipage}{\textwidth}
%			\vspace{-2cm}
%			\noindent \includegraphics[scale=0.85]{img/lzfd_logo.png} \hfill   \includegraphics[scale=0.2]{img/Logo_DHBW.pdf}
%	\end{minipage}
%	\vspace{1em}
%	\sffamily
%	\begin{center}
%		\textsf{\textbf{\Large{}Projektarbeit 1}}\\[12mm]
%		\textsf{\large{}Duale Hochschule Baden-W\"urttemberg Karlsruhe}\\[2em]
%		\textsf{\textbf{\LARGE\DerTitelDerArbeit}}\\[1.5cm]
%		\textsf{\textbf{\Large{}Studiengang Informatik}\\[3mm]} 
%		\textsf{\textbf{\Huge{}\textcolor{red}{Sperrvermerk}}}
%		\vspace{2.5em} 
%		\vspace{3em}
%		\textsf{\Large{}}
%	\vfill
%	
%	\begin{minipage}{\textwidth}
%	
%	\begin{tabbing}
%		Wissenschaftlicher Betreuer: \hspace{0.85cm}\=\kill
%		Verfasser: \> \DerAutorDerArbeit \\[1.5mm]
%		Matrikelnummer: \> 5454556 \\[1.5mm]
%		Behörde: \> \DerNameDerFirma  \\[1.5mm]
%		Abteilung: \> Landeszentrum für Datenverarbeitung\\[1.5mm]
%		Kurs: \> \DieKursbezeichnung \\[1.5mm]
%		Studiengangsleiter: \> Prof. Dr. Sebastian Ritterbusch\\[1.5mm]
%		%Wissenschaftlicher Betreuer: \> Dr. rer. nat Martin Härterich \\[1.5mm]
%		Unternehmensbetreuer: \> Sven Binnig\\[1.5mm]
%		Abgabetermin: \> 15.09.2025
%	\end{tabbing}
%	\end{minipage}
%	
%	\end{center}
%	
%\end{titlepage}

\begin{titlepage}
	\begin{minipage}{\textwidth}
			\vspace{-2cm}
			\noindent \includegraphics[scale=0.85]{img/lzfd_logo.png} \hfill   \includegraphics[scale=0.2]{img/Logo_DHBW.pdf}
	\end{minipage}
	\vspace{0.5em}
	\sffamily
	\begin{center}
		\textsf{\textbf{\Large Projektarbeit 1}}\\[0.8cm]
		\textsf{\textbf{\LARGE\DerTitelDerArbeit}} \\[0.8cm]
		\textsf{\large Fakultät Technik}\\[0.3cm]
		\textsf{\large Studiengang \Studiengang}\\[0.3cm]
		\textsf{\large Im Rahmen der Prüfung zum Bachelor of Science}\\[0.3cm]
		\textsf{\large von}\\[0.3cm]
		\textsf{\large\bfseries \DerAutorDerArbeit}\\[0.3cm]
		\textsf{\large Abgabedatum \AbgabeDatum} \\[0.8cm]
		\textsf{\textbf{\Large{}\textcolor{red}{Sperrvermerk}}}\\[0.8cm]
		%\textsf{\textbf{\Large{}Studiengang Informatik}\\[2.5cm]}
		%\textsf{\large{Duale Hochschule Baden-W\"urttemberg\\[1.5mm] Karlsruhe}}\\[2em]
		%\vspace{2.5em} 
		%\vspace{3em}
		%\textsf{\Large{}}
	\vfill
	
	\begin{minipage}{\textwidth}
	
	\begin{tabbing}
		Linke Seite \hspace{4.5cm}\=\kill
		Matrikelnummer: \> 5810399 \\[0.5mm]
		Kurs: \> \DieKursbezeichnung \\[0.5mm]
		Bearbeitungszeitraum: \> 13 Wochen \\[0.5mm] 
		Behörde: \> \DerNameDerFirma  \\[0.5mm]
		Abteilung: \> Landeszentrum für Datenverarbeitung \\[0.5mm]
		Betrieblicher Betreuer: \> Sven Binnig \\[0.5mm]
		Studiengangsleitung: \> Prof. Dr. Sebastian Ritterbusch
	\end{tabbing}
	\end{minipage}
	
	\end{center}
	
	\end{titlepage}



%--------------------------------
% Verzeichnisse - nicht benötige Verzeichnisse bitte auskommentieren / löschen.
%--------------------------------
\pagenumbering{gobble}
% Ehrenwörtliche Erklärung ewerkl.tex einziehen
\input{erklaerung}

%   Sperrvermerk
\input{sperrvermerk}

%	Kurzfassung
%\renewcommand{\abstractname}{Abstract} % Veränderter Name für das Abstract
\chapter*{Abstract}
\begin{addmargin}[1.5cm]{1.5cm}        % Erhöhte Ränder, für Abstract Look
\thispagestyle{plain}                  % Seitenzahl auf der Abstract Seite

\begin{center}
\small\textit{- Deutsch -}             % Angabe der Sprache für das Abstract
\end{center}

\vspace{0.25cm}

Das Landeszentrum für Datenverarbeitung betreibt, neben anderen IT-Dienst\-leis\-tun\-gen, ein Schwachstellen- und Bedrohungsmanagement. Für die Bereitstellung der Dienstleistungen betreibt das Landeszentrum für Datenverabeitung eine umfangreiche Infrastruktur, weshalb das Schwachstellenmanagement mithilfe einer Anwendung entlastet werden soll.\\
In der Arbeit werden zunächst Grundlagen der Informationssicherheit und Anwendungsentwicklung beschrieben. In Eigenrecherche werden potenzielle Metriken erarbeitet, die genutzt werden können, um das Schwachstellenmanagement zu entlasten. Die Metriken werden den betroffenen Fachbereichen präsentiert und im Anschluss werden Anforderungen an die Anwendung durch eine Anforderungsanalyse ermittelt. Die gestellten Anforderungen fließen in das Konzept ein, das zur Erstellung eines Prototyps verwendet wird. Der Prototyp wird in Python implementiert und verwendet unter anderen Nmap zur Informationsbeschaffung und MariaDB als relationales Datenbank Managementsystem. In der Evaluation werden die Ergebnisse des Prototyps in einer Testumgebung aufgezeigt und untersucht.\\
Die durchgeführten Tests zeigen die Funktionalität des Prototyps und den Nutzen der Anwendung. Die entwickelte Anwendung sorgt für eine Reduktion des Aufwands, den das Schwachstellenmanagement leisten muss und verbessert die Effektivität. Neben den ermittelten Metriken, die zur Verbesserung der Sicherheit der Infrastruktur genutzt werden können, liefert die Anwendung Möglichkeiten für das Berichtswesen und Analyse.

\end{addmargin}

\clearpage

\begin{addmargin}[1.5cm]{1.5cm}        % Erhöhte Ränder, für Abstract Look
\thispagestyle{plain}   
\begin{center}
\small\textit{- English -}             % Angabe der Sprache für das Abstract
\end{center}

\vspace{0.25cm}

In addition to other information technology services, the Landeszentrum für Datenverarbeitung operates a vulnerability and threat management. To provide these services, the Landeszentrum für Datenverarbeitung operates an extensive infrastructure, which is why vulnerability management is to be relieved with the help of an application.\\
The paper first describes the basics of information security and application development. In self-research, potential metrics were developed that can be used to relieve the vulnerability management. The metrics were presented to the concerned departments and requirements for the application were then identified and analyzed through an requirements analysis. The requirements provided are incorporated into the concept, which is used to create a prototype. The prototype is implemented in Python and uses, among others, Nmap for information retrieval and MariaDB as relational database management system. In the evaluation, the results of the prototype were shown and examined in a test environment.
The tests performed show the functionality of the prototype and the usefulness of the application. The developed application provides reduction of the effort the vulnerability management has to accomplish and improves the effectiveness. In addition to the identified metrics that can be used to improve the security of the infrastructure, the application provides reporting and analysis capabilities.

\end{addmargin}
\pagenumbering{Roman} % Römische Seitennummerierung
\normalfont
%	Inhaltsverzeichnis
\tableofcontents

\clearpage
\ohead{Vorwort}
\clearpage
\chapter*{Vorwort}
\addcontentsline{toc}{chapter}{Vorwort}

Zur besseren Lesbarkeit sind häufig verwendete Abkürzungen und organisationsspezifische Bezeichnungen in kursiver Schrift dargestellt und im Abkürzungsverzeichnis aufgeführt.
%	Vorwort
%\clearpage
\chapter*{Vorwort}
\addcontentsline{toc}{chapter}{Vorwort}

Zur besseren Lesbarkeit sind häufig verwendete Abkürzungen und organisationsspezifische Bezeichnungen in kursiver Schrift dargestellt und im Abkürzungsverzeichnis aufgeführt.

%	Abbildungsverzeichnis
\listoffigures

%	Tabellenverzeichnis
%\listoftables

%	Listingsverzeichnis
%\lstlistoflistings

% 	Algorithmenverzeichnis
%\listofalgorithms

% 	Abkürzungsverzeichnis (siehe Datei acronyms.tex!)
\clearpage
\chapter*{Abkürzungsverzeichnis}	
\addcontentsline{toc}{chapter}{Abkürzungsverzeichnis}

\begin{acronym}[LONGEXPO]
	\acro{ae}[AE]{Anwendungsentwicklung}
    \acro{edv}[EDV]{Elektronische Datenverarbeitung}
    \acroplural{edv}[EDV-Abteilung]{Elektronische Datenverarbeitungs-Abteilung}
    \acro{fmbw}[FM BW]{Ministerium für Finanzen Baden-Württemberg}
	\acro{lzfd}[LZfD]{Landeszentrum für Datenverarbeitung}
	\acro{ofdbw}[OFD BW]{Oberfinanzdirektion Baden-Württemberg}
	\acro{pa}[PA]{Projektarbeit}
    \acro{xml}[XML]{Extensible Markup Language}
    \acro{jaxb}[JAXB]{Java Architecture for XML Binding}
    \acro{dom}[DOM]{Document Object Model}
    \acro{sitif}[SITiF]{Sicherheitszentrum IT in der Finanzverwaltung Baden-Württemberg}
    \acro{steuerid}[Steuer-ID]{Steuerliche Identifikationsnummer}
    \acro{bzw}[bzw.]{beziehungsweise}
    \acro{csv}[CSV]{Comma-Separated Values}
    \acro{dto}[DTO]{Data Transfer Objects}
    
\end{acronym}
 
\ohead{Acronyms} % Neue Header-Definition

% Glossar
%\clearpage
\chapter*{Glossar}
\addcontentsline{toc}{chapter}{Glossar}

\begin{longtable}{p{4cm} p{10cm}}
    \caption{Glossar relevanter Begriffe} \label{tab:glossar} \\
    \toprule
    \textbf{Begriff} & \textbf{Erklärung} \\
    \midrule
    \endfirsthead

    \multicolumn{2}{c}%
    {{\bfseries \tablename\ \thetable{} -- Fortsetzung}} \\
    \toprule
    \textbf{Begriff} & \textbf{Erklärung} \\
    \midrule
    \endhead

    \midrule \multicolumn{2}{r}{{Fortsetzung auf nächster Seite}} \\
    \endfoot

    \bottomrule
    \endlastfoot

    \linkdest{arithMittel}{Arithmetisches Mittel} & Durchschnittswert eines Datensatzes, berechnet aus der Summe aller Werte geteilt durch deren Anzahl. \\
    \linkdest{cobol}{COBOL} & Programmiersprache die speziell für die Datenverarbeitung in einem Unternehmenskontext geeignet ist \cite[Vgl.][]{ibmCobol}. \\
    \linkdest{csv}{CSV} & Textdateiformat, das Tabellen oder Listen in einer einfachen Struktur speichert \cite[Vgl.][]{csv}. \\
    \linkdest{datenintegrität}{Datenintegrität} & Bezeichnet einen zu jedem Zeitpunkt korrekten, vollständigen und konsistenten Datenzustand \cite[Vgl.][]{ibmIntegritaet}. \\
    \linkdest{dom}{\acl{dom}} & Ist eine definierte Schnittstelle in der Frontendentwicklung. Sie ist plattform- und programmiersprachenunabhängig und erlaubt sowohl die Struktur als auch das Layout eines Dokuments zu verändern \cite[Vgl.][]{dom-ionos}. \\
    \linkdest{eibe}{EiBE} & \ac{konsens}-Fachverfahren zur Eingabe, Bearbeitung und Erstellung von Freistellungsbescheinigungen für Bauleistungen.  \\
    \linkdest{elfe}{ELFE-WertBV} & Wird im Rahmen von Erbschaften / Schenkungen betriebliches Vermögen übergeben, muss auf den Stichtag der Erbschaft / Schenkung eine Wertfeststellung des Betriebsvermögen durchgeführt werden. Diese Feststellung fließt als Grundlagenbescheid in den Erb-/Schenk-Bescheid ein. \\
    \linkdest{entität}{Entität} & Im Umfeld der Informationstechnik ist eine Entität laut dem Duden eine \glqq eindeutig identifizierbare Größe (z.\,B. Person oder Objekt), über die Informationen gespeichert werden\grqq\ \cite[][]{entität-duden}. \\
    \linkdest{flowchart}{Flowchart} & Ein Flowchart auch Flussdiagramm genannt, \glqq [\dots] stellt in Form eines Diagramms die Phasen eines Prozesses, Workflows, Computerprogramms oder Systems dar.\grqq\: \cite[][]{FlowChartIBM} \\
    \linkdest{github}{Github} &  Eine auf der Open-Source-Software Git basierende Plattform zum Speichern und Teilen sowie für die Versionsverwaltung von Quellcode. \cite[Vgl.][]{WasistGitHub} \\
    \linkdest{median}{Median} & Zentraler Wert einer der Größe nach sortierten Datenreihe, teilt die Daten in zwei gleich große Hälften. \\
    \linkdest{microservices}{Microservices} & Architekturmuster, bei dem komplexe Anwendungen aus unabhängigen, eigenständig laufenden Prozessen bestehen, die über sprachenunabhängige Schnittstellen miteinander kommunizieren \cite[Vgl.][]{awsMicroServices}. \\
    \linkdest{mockups}{Mockups} & Vorläufige Darstellungen oder Entwürfe, die visualisieren, wie ein System oder eine Oberfläche künftig aussehen könnte \cite[Vgl.][]{CanvaMockup}. \\
    \linkdest{mvc}{\acl{mvc}} & Aufteilung einer Software in drei zentrale Komponenten. Diese kann sowohl als Architektur- als auch Entwurfsmuster verwendet werden \cite[Vgl.][]{deacon2009model}. \\
    \linkdest{normalisierung}{Normalisierung} & Beschreibt eine Vorgehensweise für relationale Datenbanken. Mit dieser Vorgehensweise können redundante Speicherung von Informationen und daraus resultierende Inkonsistenz und Anomalien vermieden werden \cite[Vgl.][S.~82\,f.]{NormalisierungTaubner}. \\
    \linkdest{quartil}{Quartil} & Unterteilung eines Datensatzes in vier gleich große Bereiche, es gibt drei Quartile: das erste (25\%), zweite (50\%) und dritte (75\%) Quartil. \\
    \linkdest{schiefe}{Links- / Rechtsschief} & Bei schiefen Verteilungen liegt der Großteil der Daten auf einer Seite des Mittelwerts. Linksschief bedeutet Häufung rechts, rechtsschief entsprechend links. \\
    \linkdest{standardabweichung}{Standardabweichung} & Gibt an, wie stark einzelne Werte im Durchschnitt vom Mittelwert abweichen. \\
    \linkdest{varianz}{Varianz} & Maß für die durchschnittliche quadratische Abweichung der Werte vom Mittelwert. \\
    \linkdest{whisker}{Whisker} & Wertebereiche außerhalb der Box eines Boxplots. Sie geben an, wie weit die restlichen Daten nach oben und unten streuen, bevor die Ausreißer beginnen. \\
    \linkdest{wozuPapier}{WoZU Papier} & Mit dem Kontaktformular schicken Bürgerinnen und Bürger verschlüsselte Nachrichten mit Anhängen sicher und schnell an das Finanzamt. Dient der Entschlüsselung, Virenüberprüfung und Übergabe der Nachrichten mit Anhängen an WoZu Papier für die elektronische Ablage und Weiterverarbeitung in der Finanzämtern Baden-Württembergs. \\
    \linkdest{ueas}{Überwachung Außen Steuerfälle} & Stellt maschinelle Funktionen bereit, Auslandsengagements inländischer Steuerpflichtiger zu verwalten und zu überwachen.\\

\end{longtable}


%\ohead{Glossar}

%-------------------------s-------
% Start des Textteils der Arbeit
%--------------------------------
\clearpage
\ihead{\chaptername~\thechapter} % Neue Header-Definition (inner header)
\ohead{\headmark} % Neue Header-Definition (outer header)
\pagenumbering{arabic}  % Arabische Seitenzahlen

%	Einleitung
\chapter{Einleitung} \label{chap:Einleitung}
Dieses Kapitel dient dem Einstieg in die vorliegende Projektarbeit. 
Es erläutert das fachliche sowie organisatorische Umfeld, die Motivation für die Themenwahl und beschreibt die Zielsetzung der Arbeit.  


\section{Betriebliches Umfeld} \label{sec:AufbauFunktion}
%Im folgenden Kapitel werden zuerst die Aufgaben und Zuständigkeiten der \ac{ofdbw} erläutert, im Anschluss wird die Funktion des \ac{lzfd} beschrieben.

%\subsection{Aufgaben und Zuständigkeit der Oberfinanzdirektion Baden-Württemberg}
Die \ac{ofdbw} ist die Landesmittelbehörde der Steuerverwaltung. 
Sie nimmt die Dienst- und Fachaufsicht über alle 65 Finanzämter in Baden-Württemberg wahr und koordiniert deren organisatorische und fachliche Ausrichtung.
Zusätzlich ist sie für die Landesoberkasse zuständig sowie für die staatlichen Hochbauämter, die im Bereich des Bundesbaus tätig sind.
Ein weiterer Aufgabenbereich umfasst die Steuerung von Organisation, Personal, Haushalt innerhalb der Steuerverwaltung.
Die Behörde spielt außerdem eine wichtige Rolle bei der Weiterentwicklung und Digitalisierung von Prozessen, Standards und IT-Systemen in der Finanzverwaltung \cite{ofdBW2025}.

%\subsection{Das Landeszentrum für Datenverarbeitung (LZfD)}
Das \ac{lzfd} ist eine Abteilung innerhalb der \ac{ofdbw} und fungiert als zentraler IT-Dienstleister der Steuerverwaltung in Baden-Württemberg.
Es betreibt und betreut eine Vielzahl von IT-Anwendungen, die in den Finanzämtern eingesetzt werden, 
etwa Systeme zur Steuerveranlagung oder zur elektronischen Kommunikation mit Bürgerinnen und Bürgern.
Darüber hinaus unterstützt das \ac{lzfd} Fachbereiche bei der Umsetzung neuer Softwarelösungen, 
stellt die technische Infrastruktur bereit und ist für Themen IT-Sicherheit und Datenpflege zuständig \cite{lzfd2025}.
Das \ac{lzfd} ist in sieben verschiedene EDV Abteilungen aufgeteilt.

\begin{itemize}
    \item EDV 1 Zentrale Dienste, Querschnittsaufgaben
    \item EDV 2 \ac{ae}
    \item EDV 3 Applikationsmanagement
    \item EDV 4 Systembetrieb und Basisdienste 
    \item EDV 5 Service
    \item EDV 6 KONSENS, Architektur Projekte und Test 
    \item EDV 7 \ac{sitif}
\end{itemize}
\bigbreak
Die Abteilung EDV 2 der Oberfinanzdirektion Baden-Württemberg entwickelt und betreut IT-Anwendungen für die Finanzämter des Landes. 
Diese unterstützen die Verwaltung bei der korrekten Erfassung, Verarbeitung und Auswertung von Steuerdaten zur effizienten Erfüllung steuerlicher Aufgaben. 
Ein zentrales Projekt ist das länderübergreifende Vorhaben KONSENS, das eine einheitliche Softwarelösung zur Optimierung der Arbeitsprozesse in den Finanzämtern bereitstellt \cite{konsens2025}.

Die Unterabteilung EDV 211 entwickelt und pflegt die Software zur Bearbeitung der Grundsteuererklärungen nach den aktuellen gesetzlichen Vorgaben und sorgt für eine reibungslose Einführung der neuen Regelungen \cite{grundsteuerneu2025}.
Zudem passt EDV 211 die Anwendungen konstant an geänderte rechtliche Anforderungen an und arbeitet eng mit anderen IT-Abteilungen, insbesondere im Bereich IT-Sicherheit, 
zusammen, um den Schutz sensibler Daten und die Systemstabilität zu gewährleisten.

\section{Motivation}

Mit der Einführung der neuen Grundsteuerregelung in Deutschland müssen zahlreiche Grundsteuererklärungen digital erfasst und verarbeitet werden. 
Die dafür eingesetzte Software muss zuverlässig arbeiten und in der Lage sein, eine große Bandbreite unterschiedlicher Eingabekombinationen korrekt zu verarbeiten.
Die bisherige manuelle Erstellung von Testfällen ist in diesem Zusammenhang sehr zeitaufwendig und risikoanfällig, so dass wichtige Szenarien unberücksichtigt bleiben. 
Bereits geringe Abweichungen in der Berechnung können zu fehlerhaften Steuerbescheiden führen und damit zusätzlichen Aufwand für Verwaltung sowie Bürgerinnen und Bürgern verursachen.
Ein automatisierter Testfallgenerator stellt hierfür eine effiziente Lösung dar, um die Software ausreichend zu prüfen und Fehler bei den Berechnungen frühzeitig zu erkennen. 
Er kann in kurzer Zeit eine Vielzahl von Testfällen erzeugen.
Auf diese Weise lassen sich Fehler frühzeitig erkennen, die Testabdeckung erhöhen und der gesamte Testprozess deutlich beschleunigen.

\section{Zielsetzung der Arbeit}

In dieser Projektarbeit wird eine Software entwickelt, die automatisch Testfälle erstellt um die korrekte Verarbeitung von Steuererklärungen zu überprüfen. 
Oft werden solche Testfälle noch manuell angelegt. Dies ist zeitaufwendig, erfordert spezielles Fachwissen und kann zu Fehlern führen. 
Die geplante Lösung soll diesen Prozess vereinfachen und beschleunigen.
Die Software nutzt strukturierte Eingabedaten, zum Beispiel aus \ac{xml} oder \hypertarget{importantword}{Excel Dateien}\footnote{Im Folgenden wird häufig von „\hyperlink{importantword}{Excel Dateien}“ gesprochen. Damit sind Dateien gemeint, die mit Microsoft Excel verarbeitet werden können, insbesondere die Formate .xlsx (Excel-Arbeitsmappen) sowie .csv (Comma-Separated Values). 
Diese Definition umfasst sowohl native Excel-Dateien als auch tabellarische Textdateien, die häufig für den Datenaustausch verwendet werden.}. Diese enthalten alle wichtigen Informationen, erwartete Ausgaben und besondere Bedingungen. 
Das Programm liest die Daten ein, prüft sie auf Vollständigkeit und wandelt sie in ein einheitliches Format um. Dieses Format kann direkt in gängigen Testumgebungen verwendet werden, ohne dass zusätzliche Anpassungen nötig sind.
Ein Schwerpunkt liegt auf der einfachen Bedienung. 
Die Anwendung soll auch von Personen genutzt werden können, die keine Programmierkenntnisse haben. 
Über definierte Schnittstellen, wie etwa eine API, lässt sich das Programm problemlos in bestehende Testsysteme einbinden und nahtlos in vorhandene Arbeitsabläufe integrieren.

Ziel ist es, den Aufwand für die Testfallerstellung deutlich zu verringern, den Testprozess effizienter zu gestalten und die Qualität der Steuerungssoftware langfristig zu sichern. 
Durch die Automatisierung werden Fehlerquellen reduziert, Entwicklungszeiten verkürzt und die Testergebnisse zuverlässiger.


\bigbreak



%	Grundlagen
\chapter{Grundlagen} \label{chap:Grundlagen}
In diesem Kapitel werden die notwendigen technischen Vorkenntnisse vermittelt, um ein besseres Verständis späterer Schritte zu ermöglichen.

\section{Java}
Java ist eine objektorientierte Programmiersprache, die auf der sogenannten Java Virtual Machine (JVM) ausgeführt wird \cite{rentrop2017java}.
Dies ermöglicht es, Java-Programme plattformunabhängig zu nutzen, das heißt, sie laufen auf verschiedenen Betriebssystemen wie Windows, Linux oder macOS ohne Änderungen am Programmcode \cite{innowise2024java}.
Diese Plattformunabhängigkeit ist einer der Hauptvorteile von Java \cite{rentrop2017java}.

Die Programmiersprache Java verfügt über eine umfangreiche Standardbibliothek, die viele wichtige Funktionen bereitstellt, beispielsweise für die Datenverarbeitung, 
die Netzwerkkommunikation oder die Benutzeroberflächenentwicklung \cite{ullenboom2015java8}. Java ist so aufgebaut, dass Programme in Klassen und Objekten organisiert werden.
Dieses Prinzip der Objektorientierung erleichtert die Entwicklung von gut strukturierten, wartbaren und erweiterbaren Anwendungen.
Durch Konzepte wie Vererbung und Schnittstellen können Programmteile wiederverwendet und flexibel gestaltet werden \cite{innowise2024java}.
Java bietet viele Möglichkeiten zur Verarbeitung verschiedener Datenformate wie XML und Excel \cite{innowise2024java}.
%Für XML Daten wird häufig die Technologie \ac{jaxb} verwendet.
%Sie ermöglicht es, XML Dateien leicht in Java Objekte umzuwandeln und umgekehrt. Dadurch wird die Arbeit mit strukturierten Daten deutlich vereinfacht \cite{horn2025jaxb}.
%Für Excel Dateien eignet sich die Bibliothek Apache POI. 
%Sie bietet Funktionen zum Lesen und Schreiben von Tabellen und unterstützt sowohl ältere (.xls) als auch neuere (.xlsx) Formate.
%Mit diesen Werkzeugen lassen sich tabellarische und strukturierte Daten problemlos in Java-Anwendungen integrieren \cite{codeurjava2025poi}.
Dank der Plattformunabhängigkeit, der klaren Struktur und der großen Auswahl an Bibliotheken eignet sich Java besonders gut für die Entwicklung stabiler und flexibler Programme \cite{rentrop2017java} \cite{ullenboom2015java8}.
Die aktive Entwickler-Community sorgt zudem dafür, dass Java regelmäßig weiterentwickelt und an neue Anforderungen angepasst wird \cite{innowise2024java}.

\section{XML}
%\ac{xml} ist eine textbasierte Sprache zur strukturierten Speicherung und zum Austausch von Daten zwischen verschiedenen Systemen \cite{ausbildung2025xml}.
%%Sie ist unabhängig von Betriebssystemen und Programmiersprachen, was die Kommunikation zwischen unterschiedlichen Anwendungen erleichtert.
%\ac{xml} Daten werden in Elementen dargestellt, die durch Tags gekennzeichnet sind zum Beispiel <name>...</name>. 
%%Jedes Element besteht aus einem öffnenden und einem schließenden Tag zum Beispiel <name>...</name> und kann Text oder weitere Elemente enthalten \cite{wende2025xml}.
%Durch diese Verschachtelung entsteht eine klare hierarchische Struktur \cite{microsoft2025xml}.
%
%Ein \ac{xml} Dokument beginnt mit einer Deklaration, die Informationen zur \ac{xml} Version und Zeichencodierung enthält \cite{microsoft2025xml}.
%\ac{xml} Daten können sowohl in Elementen als auch in Attributen gespeichert werden. 
%Elemente eignen sich gut für umfangreiche oder hierarchisch strukturierte Informationen, da sie selbst wieder weitere Unterelemente enthalten können. 
%Attribute hingegen speichern einfache, meist metadaten ähnliche Informationen in Form von Schlüssel-Wert-Paaren direkt im öffnenden Tag, zum Beispiel <person alter="30\textquotedbl{}>Max</person>. 
%Ein Vorteil von Attributen ist, dass sie den Aufbau kompakter machen, jedoch können sie keine komplexen Datenstrukturen abbilden. 
%%Daher sollte man Attribute eher für Eigenschaften oder Metainformationen nutzen und die eigentlichen Daten in Elementen speichern \cite{jaxb}.
%%Darauf folgt ein zentrales Wurzelelement, das alle weiteren Daten umschließt.
%%Zusätzlich zu Elementen können Attribute verwendet werden, um weitere Informationen direkt in einem Element zu speichern \cite{wende2025xml}.
%Ein wichtiger Vorteil von \ac{xml} ist seine Selbstbeschreibbarkeit. Die Tags geben Auskunft darüber, welche Daten enthalten sind, sodass Menschen und Maschinen sie gut verstehen können \cite{ausbildung2025xml}.
%
%%\ac{xml} wird von nahezu allen Programmiersprachen und vielen Tools unterstützt, wodurch es sehr flexibel ist und einen großen Einsatzbereich ermöglicht.
%%Aufgrund seiner Flexibilität und der Möglichkeit, komplexe Daten genau zu strukturieren und zu prüfen, wird \ac{xml} in vielen Bereichen eingesetzt wie zum Beispiel bei Webservices, Konfigurationsdateien oder im Dokumentenaustausch \cite{wende2025xml}.
%
%\ac{xml} wird von nahezu allen Programmiersprachen und vielen Anwendungen unterstützt, dadurch wird eine hohe Flexibilität gewährleistet und den Einsatz in Bereichen wie Webservices, Konfigurationsdateien oder dem Dokumentenaustausch ermöglicht.
%
%\ac{xml} wird überwiegend als Format für die Weiterverarbeitung und Speicherung der gesammelten Daten verwendet. 
%XML zeichnet sich durch eine klare, strukturierte und hierarchische Datenorganisation aus, die es ermöglicht, Daten systematisch zu beschreiben und maschinell auszuwerten. 
%Darüber hinaus ist XML plattformunabhängig und unterstützt die Validierung durch definierte Schemata, was den sicheren Austausch und die langfristige Archivierung von Daten gewährleistet.
%XML findet vor allem als „Speicher- und Austauschformat“ Verwendung, da es durch seine strukturierte und standardisierte Form die zuverlässige Verwaltung und Übermittlung von Daten zwischen unterschiedlichen Systemen ermöglicht. 
%Die unterschiedliche Handhabung dieser Formate basiert demnach auf ihren spezifischen Stärken und Einsatzgebieten \cite{w3cXML}.

\ac{xml} ist eine textbasierte Sprache zur strukturierten Speicherung und zum Austausch von Daten zwischen verschiedenen Systemen \cite{ausbildung2025xml}. 
\ac{xml} Daten werden in Elementen dargestellt, die durch Tags gekennzeichnet sind, zum Beispiel <name>...</name>. 
Durch diese Verschachtelung entsteht eine klare hierarchische Struktur \cite{microsoft2025xml}. 
Ein \ac{xml} Dokument beginnt mit einer Deklaration, die Informationen zur \ac{xml} Version und Zeichencodierung enthält \cite{microsoft2025xml}. 
\ac{xml} Daten können sowohl in Elementen als auch in Attributen gespeichert werden. Elemente eignen sich gut für umfangreiche oder hierarchisch strukturierte Informationen, da sie selbst weitere Unterelemente enthalten können. 
Attribute hingegen speichern einfache, meist metadatenähnliche Informationen in Form von Schlüssel Wert Paaren direkt im öffnenden Tag, zum Beispiel <person alter="30">Max</person>. 
Ein Vorteil von Attributen ist, dass sie den Aufbau kompakter machen, jedoch können sie keine komplexen Datenstrukturen abbilden. Ein wichtiger Vorteil von \ac{xml} ist seine Selbstbeschreibbarkeit. 
Die Tags geben Auskunft darüber, welche Daten enthalten sind, sodass Menschen und Maschinen sie gut verstehen können \cite{ausbildung2025xml}.

\ac{xml} wird von nahezu allen Programmiersprachen und vielen Anwendungen unterstützt, wodurch eine hohe Flexibilität entsteht. 
Dies ermöglicht den Einsatz in Bereichen wie Webservices, Konfigurationsdateien oder dem Dokumentenaustausch.

Aufgrund seiner strukturierten und standardisierten Form eignet sich \ac{xml} besonders als Format für die Weiterverarbeitung und Speicherung der gesammelten Daten. 
Die klare Datenorganisation ermöglicht eine systematische Beschreibung und maschinelle Auswertung. Zusätzlich ist \ac{xml} plattformunabhängig und unterstützt die Validierung durch definierte Schemata, 
was den sicheren Austausch und die langfristige Archivierung von Daten gewährleistet. Dadurch wird \ac{xml} bevorzugt als „Speicher- und Austauschformat“ eingesetzt, 
das die zuverlässige Verwaltung und Übermittlung von Daten zwischen unterschiedlichen Systemen ermöglicht \cite{w3cXML}.


\section{Excel}
%Microsoft Excel ist ein Programm aus der Microsoft office Reihe, das vor allem dazu genutzt wird, Daten übersichtlich in Tabellen zu erfassen, zu bearbeiten und auszuwerten.
%Es gehört zur Gruppe der Tabellenkalkulationsprogramme und wird in vielen Bereichen genutzt, zum Beispiel in der Wirtschaft, der Wissenschaft oder auch im Alltag \cite{vogt2025excel}.
%Die Basis von Excel ist das Arbeitsblatt, das aus vielen kleinen Kästchen besteht, den sogenannten Zellen. 
%Diese Zellen sind in Spalten alphabetisch und Zeilen nummerisch geordnet, wodurch jede Zelle eine eindeutige Adresse erhält, wie etwa die Zelle A1.
%In den Zellen können verschiedene Inhalte eingetragen werden, zum Beispiel Zahlen, Texte oder auch Formeln \cite{vogt2025excel}.
%%Besonders hilfreich in Excel sind die automatischen Berechnungen, die eingebauten Funktionen und Formatierungen. Dadurch ist es möglich zum Beispiel mehrere Zellen miteinander zu verrechnen, 
%%die Nutzung von Funktionen wie zum Beispiel Durchschnitt MITTELWERT zu verwenden oder WENN Bedingung um Formatierungen anzupassen, 
%%sodass bestimmte Werte zum Beispiel farblich markiert werden, wenn sie eine bestimmte Bedingung erfüllen.
%In Excel sind die automatischen Berechnungen, eingebauten Funktionen und bedingte Formatierungen besonders nützlich. 
%Mit den automatischen Berechnungen können mehrere Zellen miteinander verrechnet werden, was komplexe Berechnungen vereinfacht. 
%Darüber hinaus stehen eine Vielzahl von Funktionen zur Verfügung, wie zum Beispiel MITTELWERT, um den Durchschnitt zu berechnen, oder WENN, um Bedingungen zu prüfen und daraufhin bestimmte Werte zu berechnen oder darzustellen. 
%Zusätzlich bietet Excel die Möglichkeit, bedingte Formatierungen zu verwenden. 
%Diese Funktion ermöglicht es, Zellen automatisch farblich oder anders hervorzuheben, wenn bestimmte Kriterien erfüllt sind, was die Übersichtlichkeit verbessert und die Analyse von Daten vereinfacht.
%
%
%Ein weiterer wichtiger Aspekt ist die Zellformatierung und das Datenformat. 
%Excel unterscheidet verschiedene Datentypen wie Zahlen, Texte, Datumsangaben, Uhrzeiten, Prozentwerte oder Währungen \cite{vogt2025excel}.
%Diese Formate beeinflussen sowohl die Darstellung als auch das Verhalten der Daten, zum Beispiel bei Berechnungen oder beim Sortieren \cite{microsoft2025pivot}.
%So kann Excel automatisch erkennen, ob es sich bei einer Eingabe um eine Zahl oder ein Datum handelt, und die Zelle entsprechend formatieren.
%Darüber hinaus bietet Excel viele Werkzeuge zur Analyse und Visualisierung von Daten, wie Filter, Diagramme, Pivot-Tabellen oder Datenschnitte. 
%Diese Funktionen ermöglichen es, große Datenmengen übersichtlich auszuwerten und anschaulich darzustellen \cite{microsoft2025pivot}.
%
%Excel-Dateien werden häufig zur Erfassung und Sammlung von Daten eingesetzt, da sie eine benutzerfreundliche und übersichtliche Oberfläche bieten. 
%Durch die tabellarische Darstellung können Anwender Daten einfach eingeben, bearbeiten und strukturieren, auch ohne umfangreiche technische Kenntnisse. 
%Die weite Verbreitung von Excel und die Möglichkeit, Daten direkt zu visualisieren und zu analysieren, machen es zu einem beliebten Werkzeug insbesondere für die initiale Datenerfassung.
%Excel wird somit häufig als „Eingabewerkzeug“ verwendet, da es eine einfache und flexible Möglichkeit bietet, Daten direkt durch Menschen zu erfassen und zu bearbeiten \cite{davenport2007competing}.

Microsoft Excel ist ein Programm aus der Microsoft Office Reihe, das vor allem dazu genutzt wird, Daten übersichtlich in Tabellen zu erfassen, zu bearbeiten und auszuwerten. 
Es gehört zur Gruppe der Tabellenkalkulationsprogramme und findet breite Anwendung in Bereichen wie Wirtschaft, Wissenschaft oder auch im Alltag \cite{vogt2025excel}. 
Die Grundlage von Excel bildet das Arbeitsblatt, das aus vielen kleinen Zellen besteht. 
Diese sind in Spalten (alphabetisch) und Zeilen (nummerisch) organisiert, sodass jede Zelle eine eindeutige Adresse wie A1 erhält. 
In den Zellen lassen sich verschiedene Inhalte eintragen, etwa Zahlen, Texte oder Datumsangaben \cite{vogt2025excel}.

Ein zentraler Vorteil von Excel ist die tabellarische Darstellung der Daten, die eine klare Struktur und eine einfache Orientierung ermöglicht. 
Darüber hinaus bietet Excel verschiedene Werkzeuge zur visuellen Aufbereitung und Analyse von Daten, wie zum Beispiel Filter, Diagramme, Pivot Tabellen oder bedingte Formatierungen. 
Diese helfen dabei, Muster, Auffälligkeiten oder Abweichungen in größeren Datenmengen schnell zu erkennen und gezielt auszuwerten \cite{microsoft2025pivot}.

Ein weiterer wichtiger Aspekt ist die Möglichkeit zur individuellen Zellformatierung. 
Excel unterscheidet verschiedene Datentypen wie Zahlen, Texte, Datumsangaben, Uhrzeiten, Prozentwerte oder Währungen \cite{vogt2025excel}. 
Diese Formate beeinflussen sowohl die Darstellung als auch das Verhalten der Daten, zum Beispiel beim Sortieren oder bei der Berechnung von Summen \cite{microsoft2025pivot}. 
Excel erkennt viele dieser Typen automatisch, was den Arbeitsaufwand bei der Datenerfassung reduziert.

Excel Dateien werden häufig zur Datenerfassung eingesetzt, da die benutzerfreundliche Oberfläche eine intuitive Eingabe und Strukturierung ermöglicht, auch ohne tiefgehende technische Kenntnisse. 
Die breite Verfügbarkeit, die einfache Bedienbarkeit sowie die integrierten Auswertungs und Darstellungsmöglichkeiten machen Excel zu einem beliebten Werkzeug für die initiale Datensammlung. 
Es dient daher oft als „Eingabewerkzeug“, das eine direkte und flexible Interaktion mit den Daten erlaubt \cite{davenport2007competing}.


\section{XML Verarbeitung in Java mit JAXB und DOM}


Um \ac{xml} Dateien mit Java zu verarbeiten, werden zwei Java Technologien, \ac{jaxb} und \ac{dom} benutzt, diese werden im folgenden näher erläutert.

\subsection{JAXB}

\ac{jaxb} ist eine Java Technologie, mit der sich \ac{xml} Daten direkt in Java Objekte umwandeln lassen und umgekehrt \cite{horn2025jaxb}. 
Dieser Vorgang wird Marshalling genannt, wenn Java Objekte in \ac{xml} geschrieben werden, und Unmarshalling, wenn \ac{xml} in Java Objekte eingelesen wird. 
Dadurch können \ac{xml} Daten strukturiert verarbeitet werden, ohne dass Entwickler selbst komplexe Parser schreiben müssen. Das spart Zeit und verringert den Programmieraufwand \cite{jaxb}.
Für die Umsetzung werden Java Klassen mit speziellen \ac{jaxb} Annotationen versehen, zum Beispiel @XmlElement, @XmlAttribute oder @XmlRootElement. 
Diese legen fest, wie Felder und Methoden der Klasse mit den \ac{xml} Elementen und Attributen verknüpft sind. 
Alternativ kann \ac{jaxb} aus einer \ac{xml} Schema Datei (XSD) automatisch passende Java Klassen erzeugen, was besonders bei umfangreichen oder komplexen Datenstrukturen hilfreich ist.
Ein weiterer Vorteil ist, dass der Code übersichtlich und gut wartbar bleibt, da direkt mit Java Objekten gearbeitet wird \cite{horn2025jaxb}. 
Änderungen an der \ac{xml} Struktur lassen sich durch Anpassungen an den Klassen oder am Schema schnell umsetzen \cite{jaxb}. 
\ac{jaxb} unterstützt außerdem Namespaces, komplexe Datentypen und Listen, sodass auch große und verschachtelte \ac{xml} Dokumente verarbeitet werden können.
Ein Nachteil von \ac{jaxb} ist, dass es das gesamte \ac{xml} Dokument im Speicher hält, was bei sehr großen Dateien zu einem erhöhten Speicherverbrauch führen kann.
\ac{jaxb} wird vor allem in Anwendungen eingesetzt, die regelmäßig \ac{xml} Daten lesen oder schreiben, zum Beispiel bei Webservices, Konfigurationsdateien, Datenaustausch zwischen Systemen oder standardisierten Dokumentformaten \cite{datacenterJava}. 

\subsection{DOM}
Das \ac{dom} ist ein standardisiertes Modell, mit dem \ac{xml} oder HTML Dokumente als Baumstruktur dargestellt und bearbeitet werden \cite{ausbildung2025xml}. 
Über \ac{dom} können einzelne Elemente gezielt gelesen, geändert, gelöscht oder ergänzt werden. 
Das gesamte Dokument wird dabei in den Arbeitsspeicher geladen und als Hierarchie aus Knoten dargestellt, wobei jedes Element, Attribut oder Textstück einen eigenen Knoten bildet \cite{wende2025xml}. 
So kann man gezielt auf bestimmte Teile des Dokuments zugreifen und diese bearbeiten.
Ein Vorteil ist, dass Entwickler das Dokument wie ein Objektmodell behandeln können \cite{microsoft2025xml}. 
Über Programmierschnittstellen (APIs) lassen sich Knoten suchen, verschieben, kopieren oder neu anlegen. 
\ac{dom} ist sprachunabhängig und wird in vielen Programmiersprachen wie Java, JavaScript, Python oder C\# unterstützt.
\ac{dom} wird oft genutzt, wenn komplexe Dokumente verarbeitet werden müssen, zum Beispiel bei der Anpassung von Webseiten im Browser, beim Auslesen bestimmter Daten aus \ac{xml} Dateien oder beim Erstellen neuer Dokumente aus vorhandenen Strukturen \cite{wende2025xml}. 
Es eignet sich besonders, wenn man den kompletten Überblick über die Struktur braucht oder an mehreren Stellen Änderungen vornehmen will.
Ein Nachteil ist, dass immer das gesamte Dokument im Speicher gehalten wird. 
Bei sehr großen Dateien kann das viel Speicher verbrauchen und zu längeren Ladezeiten führen. 
In solchen Fällen nutzt man oft Alternativen wie SAX (Simple API for XML) oder StAX (Streaming API for XML), die das Dokument Schritt für Schritt verarbeiten und weniger Speicher benötigen \cite{ausbildung2025xml}.
Trotzdem bleibt \ac{dom} wegen seiner klaren Struktur, der einfachen Navigation und der breiten Unterstützung ein wichtiges Werkzeug für die Arbeit mit \ac{xml} und HTML Dokumenten \cite{wende2025xml}.


\section{Excel Verarbeitung in Java mit Apache POI}

Zur Verarbeitung von Excel-Dateien wurde die Java-Bibliothek Apache POI verwendet. 
Diese frei verfügbare Bibliothek ermöglicht es, Microsoft-Office-Dokumente, vor allem Excel Dateien im Format .xls und .xlsx direkt im Programm zu öffnen, auszulesen, zu bearbeiten und auch neue Dateien zu erstellen \cite{apachePOI}.

Mit Apache POI kann gezielt auf Inhalte von Tabellen zugegriffen, einzelne Zellen oder ganze Bereiche verändert und neue Arbeitsblätter angelegt werden
Ergebnisse, die während der Ausführung entstehen, lassen sich problemlos wieder in eine Excel Datei schreiben, etwa zur Dokumentation oder für spätere Auswertungen \cite{apachePOI}\cite{baeldung2025apachepoi}.
Die Bibliothek besteht aus verschiedenen Teilen. HSSF (Horrible Spreadsheet Format) ist zuständig für ältere .xls Dateien, während XSSF (XML Spreadsheet Format) mit dem neueren .xlsx Format arbeitet. 
Darüber hinaus bietet Apache POI auch Funktionen, mit denen man Inhalte formatieren, Formeln einfügen oder sogar Diagramme erstellen kann, ganz ohne Excel manuell zu öffnen \cite{baeldung2025apachepoi}.
Durch den Einsatz von Apache POI lässt sich die Arbeit mit Excel-Dateien vollständig automatisieren. Das spart Zeit, reduziert Fehler und macht Testabläufe einfacher und zuverlässiger. 
Außerdem können Excel-Daten direkt in Java-Anwendungen eingebunden werden, was die Entwicklung deutlich erleichtert \cite{apachePOI,baeldung2025apachepoi}.


%\section{JUnit Tests}
%JUnit ist ein Framework, das speziell für Java entwickelt wurde und es ermöglicht, automatische Tests für einzelne Teile des Codes zu schreiben \cite{junit2025overview}.
%JUnit Tests sind ein wichtiger Bestandteil in der Softwareentwicklung, wodurch die Qualität und Korrektheit der Programme sichergestellt wird. 
%So können Entwickler prüfen, ob einzelne Methoden oder Komponenten wie gewollt arbeiten \cite{baeldung2024junit}.
%Mit JUnit werden spezielle Testfälle definiert, bei denen bestimmte Eingabewerte an eine Methode übergeben werden. 
%Anschließend wird überprüft, ob das Ergebnis den Erwartungen entspricht oder ob auftretende Fehler korrekt behandelt werden \cite{gamma2005junit}. .
%Dabei kennzeichnen einfache Annotationen wie @Test Methoden als Testfälle und ermöglichen es dem Test Framework, diese automatisch zu erkennen und auszuführen, 
%so wird der Code übersichtlich strukturiert und der Testablauf organisiert \cite{junit2025overview}.
%Ein großer Vorteil von JUnit Tests ist, dass Fehler früh in der Entwicklungsphase erkannt und behoben werden können. Tests können immer wieder automatisch ausgeführt werden, zum Beispiel bei jeder Änderung im Programm. 
%So wird sichergestellt, dass neue Anpassungen keine Probleme verursachen. Das erhöht die Sicherheit und Verlässlichkeit des Codes \cite{junit2025overview}\cite{baeldung2024junit}.

%\section{Aufbau und Struktur von Testfalldaten}
%Die Testfalldaten sind in einem klaren Schema aufgebaut, um eine konsistente und nachvollziehbare Verarbeitung zu ermöglichen. 
%Jeder Testfall enthält eine laufende Nummer zur eindeutigen Identifizierung sowie Angaben zu den Eingabewerten, 
%den erwarteten Ergebnissen und eventuell eine kurze Beschreibung. Diese Struktur erleichtert sowohl die manuelle Nachvollziehbarkeit als auch die automatische Weiterverarbeitung im Testfallgenerator.
%Durch die einheitliche Gliederung können Fehler leichter erkannt werden und die Vergleichbarkeit der Testfälle bleibt gewährleistet.





%\chapter{Anforderungen} \label{chap:Anforderungen}
Dieses Kapitel behandelt die Ergebnisse der Anforderungserhebung, 
bestimmt die Anforderungen an den Testfallgenerator und stellt geeignete Lösungsansätze zur Umsetzung dieser Anforderungen vor.

\section{Anforderungserhebung}
%Bevor mit der Umsetzung des Testfallgenerators begonnen werden konnte, mussten vorab die Anforderungen geklärt werden.
%%Dazu wurde mit den beteiligten Kollegen besprochen, was das Tool können soll, welche Eingabeformate unterstützt werden müssen und wie die Testfälle aufgebaut sein sollen. 
%Es wurde intern festgelegt, welche Funktionen sowie Eingabe- und Ausgabeformate unterstützt werden müssen und welche Struktur die Testfalldaten haben sollen, da diese für die interne Weiterverarbeitung genutzt werden.   
%Wichtig war auch zu verstehen, wie die Testfälle später genutzt werden und welche Regeln oder Abläufe dabei beachtet werden müssen. 
%Diese Ergebnisse bildeten die Grundlage für die technische Umsetzung.

Bevor mit der Entwicklung des Testfallgenerators begonnen wurde, mussten zunächst die Anforderungen festgelegt werden. 
Dabei wurde festgelegt, welche Funktionen der Generator später unterstützen soll. 
Außerdem wurden die erforderlichen Eingabe und Ausgabeformate bestimmt, um eine reibungslose Nutzung und Weiterverarbeitung der Testfalldaten zu gewährleisten.
Ein wichtiger Punkt war auch die Struktur der Testfalldaten. 
Diese musste so gestaltet sein, dass sie von anderen Systemen oder Anwendern problemlos weiterverarbeitet werden kann. 
Zusätzlich wurde geprüft, wie die Testfälle später verwendet werden, um sicherzustellen, dass alle notwendigen Regeln berücksichtigt werden.

Diese Erkenntnisse bildeten die Grundlage für die technische Umsetzung des Testfallgenerators. 
Durch die klare Definition der Anforderungen konnte die Entwicklung zielgerichtet und effizient erfolgen.

\section{Anforderungen an den Testfallgenerator}
Der Testfallgenerator soll automatisch Testfälle aus strukturierten Eingaben wie \ac{xml} oder Excel erzeugen können. Dabei muss er die Daten richtig einlesen, die Testfälle im gewünschten Format ausgeben und bestimmte Regeln bei der Erzeugung beachten.
Das Tool soll außerdem so gebaut sein, dass es leicht angepasst oder erweitert kann, zum Beispiel für andere Datenformate oder neue Anforderungen.
Die Eingabe- und Ausgabedatenstruktur soll klar aufgebaut und nachvollziehbar sein.

Die folgenden Anforderungen sind in funktionale und nicht funktionale Anforderungen aufgelistet.

\bigskip \textbf{Funktionale Anforderungen:}
\begin{itemize}
    \item Korrekter Datenimport der Eingabedaten
    \begin{quote}
        Der Testfallgenerator muss in der Lage sein, Testfalldaten aus Excel beziehungsweise \ac{xml} Dateien fehlerfrei zu importieren, wobei alle relevanten Ergebnisse korrekt übernommen werden
    \end{quote}
    \item Automatische Erzeugung von Testfällen basierend auf den eingelesenen Daten
    \begin{quote}
        Das Programm soll automatisch strukturierte Testfälle generieren, sobald die Eingabedaten aus der Excel beziehungsweise \ac{xml} Datei erfolgreich importiert wurden
    \end{quote}
    \item Unterstützung mehrerer Ausgabeformate
    \begin{quote}
        Der Testfallgenerator soll die erzeugten Testfälle in verschiedenen Formaten exportieren können
    \end{quote}
\end{itemize}


\bigskip \textbf{Nicht Funktionale Anforderungen:}
\begin{itemize}
    \item Effiziente Verarbeitung großer Daten
    \begin{quote}
        Der Testfallgenerator soll auch bei sehr umfangreichen Eingabedateien, eine zügige Verarbeitung gewährleisten, um die Benutzerfreundlichkeit nicht zu beeinträchtigen
    \end{quote}
    \item Zuverlässigkeit der Korrektheit der Testfälle
    \begin{quote}
        Der Testfallgenerator soll sicherstellen, dass alle erzeugten Testfälle inhaltlich korrekt und vollständig sind, sodass sie ohne manuelle Nachkorrektur in Testsystemen verwendet werden können
    \end{quote}
    \item Wartbarkeit des Generators
    \begin{quote}
        Der Quellcode des Generators soll so strukturiert, dokumentiert und modular aufgebaut sein, dass zukünftige Änderungen oder Erweiterungen mit minimalem Aufwand möglich sind
    \end{quote}
    \item Einfache Bedienung für die Anwender
    \begin{quote}
        Der Testfallgenerator soll über klar dokumentierte und verständliche Schnittstellen oder Befehle verfügen, sodass Anwender ohne großen Aufwand Testfälle importieren und exportieren können
    \end{quote}
    \item Möglichkeit zur Erweiterbarkeit neuer Datenformate
    \begin{quote}
        Der Testfallgenerator soll so gestaltet sein, dass künftig problemlos neue Eingabe und Ausgabeformate hinzugefügt werden können, ohne den bestehenden Code stark ändern zu müssen
    \end{quote}
\end{itemize}
\bigbreak
In einem Projekt ist es wichtig, die verschiedenen Anforderungen nach ihrer Bedeutung zu ordnen, dafür existieren verschiedene Methoden zur Einteilung, diese werden im Folgenden näher erläutert.

Das Kano Modell unterscheidet Anforderungen in Basisfaktoren, Leistungsfaktoren und Begeisterungsfaktoren, um zu zeigen, wie ihre Erfüllung die Zufriedenheit von Nutzern beeinflusst. 
Es macht ersichtlich, welche Merkmale als selbstverständlich erwartet werden, welche direkt die Zufriedenheit steigern und welche einen überraschend positiven Effekt haben \cite{kano1984quality}.

Die RICE Kategorisierung bildet einen nummerischen Score aus Reach × Impact × Confidence / Effort, sodass Anforderungen anhand von Reichweite, Wirkung, Vertrauenswürdigkeit der Schätzung und Aufwand vergleichbar werden. 
Die Methode erlaubt ein quantifiziertes Ranking, das hilft, viele Ideen systematisch zu ordnen \cite{perri2018escaping}.

Die WSJF Kategorisierung priorisiert nach dem Verhältnis Cost of Delay zur geschätzten Dauer, sodass Aufgaben mit dem höchsten wirtschaftlichen Nutzen pro Zeiteinheit zuerst bearbeitet werden. 
Die Methode fokussiert Time to Market und ökonomische Trade offs, indem sie Verzögerungskosten in Relation zur Aufwandsdauer setzt \cite{leffingwell2018safe}.

Die MoSCoW Methode teilt Anforderungen in vier Kategorien ein, Must (Muss), Should (Soll), Could (Kann) und Won’t (wird nicht sein). Muss Anforderungen sind unverzichtbar, ohne sie kann das Projektziel nicht erreicht werden. 
Soll Anforderungen sind ebenfalls wichtig, tragen aber eher zur Verbesserung oder Optimierung bei. Kann Anforderungen sind "nice to have" Funktionen, die umgesetzt werden, wenn Zeit und Ressourcen verfügbar sind. 
Won’t Anforderungen werden für das betrachtete Projekt bewusst ausgeschlossen \cite{agilebusinessconsortium2014dsdm}.
 
Zur Kategorisierung der Folgenden Anforderungen wird die MoSCoW Methode verwendet, diese überzeugt durch ihre einfache Struktur und klare Priorisierung, 
wodurch sie sehr verständlich ist und eine einfache, klare Grundlage für die Entscheidungen schafft.

%. Dafür unterscheidet man zwischen Muss, Soll und Kann Anforderungen. 
%Muss Anforderungen sind unverzichtbar, ohne sie kann das Projektziel nicht erreicht werden. Soll Anforderungen sind ebenfalls wichtig, tragen aber eher zur verbesserung oder Optimierung bei. Kann Anforderungen sind "nice to have" Funktionen,
%die umgesetzt werden, wenn Zeit und Ressourcen verfügbar sind. Zur Priosierung dieser Anforderungen wird häufig die MoSCoW Methode verwendet. Der Begriff steht für Must have, Should have, Could have und Won't have. 
%Diese Methode hilft den Überblick zu behalten und sicherzustellen, dass zuerst die wirklich wichtigen Punkte umgesetzt werden, bevor man sich um zusätzliche Wünsche kümmert.

Die Anforderungen sind in den Kategorien Muss-, Soll- und Kann Anforderungen zugeordnet. Eine Übersicht bietet die folgende Tabelle.

\noindent
\begin{longtable}{|p{3cm}|p{\dimexpr\textwidth-3cm-2\tabcolsep-2\arrayrulewidth\relax}|}
\hline
\textbf{Kategorie} & \textbf{Beschreibung der Anforderung} \\
\hline
\endhead
\hline
\endfoot
Muss & Korrekter Datenimport der eingegebenen Daten \\
\hline
Muss & Zuverlässigkeit der Korrektheit der Testfälle \\
\hline
Soll & Möglichkeit zur Erweiterbarkeit neuer Datenformate \\
\hline
Soll & Automatische Erzeugung von Testfällen basierend auf den eingelesenen Daten \\
\hline
Soll & Effiziente Verarbeitung großer Datenmengen \\
\hline
Soll & Wartbarkeit des Generators \\
\hline
Kann & Unterstützung mehrerer Ausgabeformate \\
\hline
Kann & Einfache Bedienung für die Anwender \\
\hline
\end{longtable}


%Zu den \textbf{Muss Anforderungen} zählt der korrekte Import der eingegebenen Daten. 
%Diese Funktion ist wesentlich, da ohne einen fehlerfreien und vollständigen Datenimport keine Testfälle erstellt werden können. 
%Ebenso unverzichtbar ist die Gewährleistung der Korrektheit der Testfälle, da nur fehlerfreie Ergebnisse eine zuverlässige Überprüfung der Logik in der Steuererklärungsoftware ermöglichen.

Zu den \textbf{Muss Anforderungen} zählt der korrekte Import der eingegebenen Daten. Diese Funktion ist wesentlich, da ohne einen fehlerfreien und vollständigen Datenimport keine Testfälle erstellt werden können. 
Ebenso unverzichtbar ist die Gewährleistung der Korrektheit der Testfälle, da nur fehlerfreie Ergebnisse eine zuverlässige Überprüfung der Logik in der Steuererklärung Software ermöglichen. 
MoSCoW klassifiziert diese Anforderungen als Must da ohne diese der Testfallgenerator seine Funktion nicht erfüllen kann.

%Zu den \textbf{Soll Anforderungen} gehört die Möglichkeit, neue Datenformate zu unterstützen, um die Software flexibel an zukünftige Anforderungen anpassen zu können, da diese Funktion für zukünftige Erweiterungen wichtig ist und für den Anfang nicht relevant ist.  
%Auch die automatische Erzeugung von Testfällen auf Basis der eingelesenen Daten ist ein wesentlicher Bestandteil, der den Prozess deutlich beschleunigt, im Bedarfsfall jedoch auch manuell umgesetzt werden könnte. 
%Die effiziente Verarbeitung großer Datenmengen trägt zu einer hohen Leistungsfähigkeit bei, ist jedoch nicht zwingend für die Grundfunktion erforderlich. 
%Die Wartbarkeit des Generators ist ebenfalls von Bedeutung, um spätere Anpassungen und Fehlerbehebungen mit geringem Aufwand vornehmen zu können.
%Die genannten Anforderungen fallen in das Sollkriterium, da der Testfallgenerator auch ohne diese Funktion einsatzfähig ist und nicht zwingend darauf angewiesen ist.

Zu den \textbf{Soll Anforderungen} gehört die Möglichkeit, neue Datenformate zu unterstützen, um die Software flexibel an zukünftige Anforderungen anpassen zu können, da diese Funktion für zukünftige Erweiterungen wichtig ist und für den Anfang nicht relevant ist. 
Auch die automatische Erzeugung von Testfällen auf Basis der eingelesenen Daten ist ein wesentlicher Bestandteil, der den Prozess deutlich beschleunigt, im Bedarfsfall jedoch auch manuell umgesetzt werden könnte. 
Die effiziente Verarbeitung großer Datenmengen trägt zu einer hohen Leistungsfähigkeit bei, ist jedoch nicht zwingend für die Grundfunktion erforderlich. 
Die Wartbarkeit des Generators ist ebenfalls von Bedeutung, um spätere Anpassungen und Fehlerbehebungen mit geringem Aufwand vornehmen zu können. 
Die MoSCoW Methode ordnet diese Anforderungen als Should ein, da sie die Qualität und Erweiterbarkeit des Testfallgenerators deutlich erhöhen aber nicht zwingend für die Grundfunktionalität erforderlich sind.

%Zu den \textbf{Kann Anforderungen} zählen die Unterstützung mehrerer Ausgabeformate, die den Anwendern zusätzliche Möglichkeiten bei der Weiterverarbeitung der Testergebnisse bietet, 
%sowie eine besonders einfache Bedienung, die den Einstieg erleichtert und die Nutzererfahrung verbessert. 
%Diese Funktionen sind nicht zwingend erforderlich, damit die Software ihre funktionalität erfüllt, sondern um den Komfort und die Flexibilität der Nutzer zu erhöhen. 
%Daher sind diese als Kann Anforderungen kategorisiert, da der Testfallgenerator ohne diese Erweiterungen einsatzfähig bleibt.

Zu den \textbf{Kann Anforderungen} zählen die Unterstützung mehrerer Ausgabeformate, die den Anwendern zusätzliche Möglichkeiten bei der Weiterverarbeitung der Testergebnisse bietet, sowie eine besonders einfache Bedienung, die den Einstieg erleichtert und die Nutzererfahrung verbessert. 
MoSCoW klassifiziert diese Anforderungen als Could, da sie den Bedienkomfort und die Flexibilität des Testfallgenerators erhöhen, für die erste funktionsfähige Version des Testfallgenerators jedoch nicht zwingend erforderlich sind.

Damit ist die Priorisierung abgeschlossen und die Anforderungen sind klar nach Wichtigkeit geordnet. Die Einordnung bildet die Grundlage für die Planung und zeigt, welche Funktionen zuerst umgesetzt werden müssen und welche bei Bedarf verschoben werden könne.
Prioritäten sollten regelmäßig übergeprüft und bei neuen Erkenntnissen oder geänderten Rahmenbedingungen angepasst werden.

%\section{Lösungsansatz zur Umsetzung der Anforderungen}
%
%Um die gestellten Anforderungen effizient umzusetzen, wurde der Testfallgenerator in der Programmiersprache Java entwickelt. 
%Das grundlegende Konzept besteht darin, unterschiedliche Eingabedatenformate, wie zum Beispiel \ac{xml} oder Excel Dateien, einzulesen und diese anschließend intern weiterzuverarbeiten. 
%Auf Basis dieser verarbeiteten Daten werden automatisch Testfälle generiert.
%
%Zuerst werden die eingelesenen Daten in Java Objekte umgewandelt, damit sie im Programm übersichtlich und gut organisiert verarbeitet werden können. 
%Danach werden die Testfälle nach festgelegten Regeln erstellt, sodass die Ergebnisse zuverlässig und nachvollziehbar sind. 
%Zum Schluss werden die fertigen Testfälle in einem passenden Format ausgegeben, das für die weiteren Schritte oder die Testumgebung verwendet werden kann.
%
%Besonders wichtig bei der Entwicklung des Testfallgenerators war der Aufbau des Codes. 
%Dieser wurde bewusst so gestaltet, dass er in Zukunft leicht erweitert und an neue Anforderungen angepasst werden kann. 
%Das ermöglicht zum Beispiel die einfache Integration neuer Datenformate oder die Anpassung an Änderungen in der Datenstruktur, ohne dass bestehende Funktionen beeinträchtigt werden.
%
%Ein zentrales Ziel der Lösung war es, eine einfache und zuverlässige Funktion sicherzustellen, die gleichzeitig flexibel genug ist, um auch künftigen Anforderungen gerecht zu werden. 
%So kann der Testfallgenerator langfristig und effizient in verschiedenen Anwendungsszenarien eingesetzt werden.
\chapter{Anforderungen} \label{chap:Anforderungen}
Dieses Kapitel behandelt die Ergebnisse der Anforderungserhebung, 
bestimmt die Anforderungen an den Testfallgenerator und stellt geeignete Lösungsansätze zur Umsetzung dieser Anforderungen vor.

\section{Anforderungserhebung}
%Bevor mit der Umsetzung des Testfallgenerators begonnen werden konnte, mussten vorab die Anforderungen geklärt werden.
%%Dazu wurde mit den beteiligten Kollegen besprochen, was das Tool können soll, welche Eingabeformate unterstützt werden müssen und wie die Testfälle aufgebaut sein sollen. 
%Es wurde intern festgelegt, welche Funktionen sowie Eingabe- und Ausgabeformate unterstützt werden müssen und welche Struktur die Testfalldaten haben sollen, da diese für die interne Weiterverarbeitung genutzt werden.   
%Wichtig war auch zu verstehen, wie die Testfälle später genutzt werden und welche Regeln oder Abläufe dabei beachtet werden müssen. 
%Diese Ergebnisse bildeten die Grundlage für die technische Umsetzung.

Bevor mit der Entwicklung des Testfallgenerators begonnen wurde, mussten zunächst die Anforderungen festgelegt werden. 
Dabei wurde festgelegt, welche Funktionen der Generator später unterstützen soll. 
Außerdem wurden die erforderlichen Eingabe und Ausgabeformate bestimmt, um eine reibungslose Nutzung und Weiterverarbeitung der Testfalldaten zu gewährleisten.
Ein wichtiger Punkt war auch die Struktur der Testfalldaten. 
Diese musste so gestaltet sein, dass sie von anderen Systemen oder Anwendern problemlos weiterverarbeitet werden kann. 
Zusätzlich wurde geprüft, wie die Testfälle später verwendet werden, um sicherzustellen, dass alle notwendigen Regeln berücksichtigt werden.

Diese Erkenntnisse bildeten die Grundlage für die technische Umsetzung des Testfallgenerators. 
Durch die klare Definition der Anforderungen konnte die Entwicklung zielgerichtet und effizient erfolgen.

\section{Anforderungen an den Testfallgenerator}
Der Testfallgenerator soll automatisch Testfälle aus strukturierten Eingaben wie \ac{xml} oder Excel erzeugen können. Dabei muss er die Daten richtig einlesen, die Testfälle im gewünschten Format ausgeben und bestimmte Regeln bei der Erzeugung beachten.
Das Tool soll außerdem so gebaut sein, dass es leicht angepasst oder erweitert kann, zum Beispiel für andere Datenformate oder neue Anforderungen.
Die Eingabe- und Ausgabedatenstruktur soll klar aufgebaut und nachvollziehbar sein.

Die folgenden Anforderungen sind in funktionale und nicht funktionale Anforderungen aufgelistet.

\bigskip \textbf{Funktionale Anforderungen:}
\begin{itemize}
    \item Korrekter Datenimport der Eingabedaten
    \begin{quote}
        Der Testfallgenerator muss in der Lage sein, Testfalldaten aus Excel beziehungsweise \ac{xml} Dateien fehlerfrei zu importieren, wobei alle relevanten Ergebnisse korrekt übernommen werden
    \end{quote}
    \item Automatische Erzeugung von Testfällen basierend auf den eingelesenen Daten
    \begin{quote}
        Das Programm soll automatisch strukturierte Testfälle generieren, sobald die Eingabedaten aus der Excel beziehungsweise \ac{xml} Datei erfolgreich importiert wurden
    \end{quote}
    \item Unterstützung mehrerer Ausgabeformate
    \begin{quote}
        Der Testfallgenerator soll die erzeugten Testfälle in verschiedenen Formaten exportieren können
    \end{quote}
\end{itemize}


\bigskip \textbf{Nicht Funktionale Anforderungen:}
\begin{itemize}
    \item Effiziente Verarbeitung großer Daten
    \begin{quote}
        Der Testfallgenerator soll auch bei sehr umfangreichen Eingabedateien, eine zügige Verarbeitung gewährleisten, um die Benutzerfreundlichkeit nicht zu beeinträchtigen
    \end{quote}
    \item Zuverlässigkeit der Korrektheit der Testfälle
    \begin{quote}
        Der Testfallgenerator soll sicherstellen, dass alle erzeugten Testfälle inhaltlich korrekt und vollständig sind, sodass sie ohne manuelle Nachkorrektur in Testsystemen verwendet werden können
    \end{quote}
    \item Wartbarkeit des Generators
    \begin{quote}
        Der Quellcode des Generators soll so strukturiert, dokumentiert und modular aufgebaut sein, dass zukünftige Änderungen oder Erweiterungen mit minimalem Aufwand möglich sind
    \end{quote}
    \item Einfache Bedienung für die Anwender
    \begin{quote}
        Der Testfallgenerator soll über klar dokumentierte und verständliche Schnittstellen oder Befehle verfügen, sodass Anwender ohne großen Aufwand Testfälle importieren und exportieren können
    \end{quote}
    \item Möglichkeit zur Erweiterbarkeit neuer Datenformate
    \begin{quote}
        Der Testfallgenerator soll so gestaltet sein, dass künftig problemlos neue Eingabe und Ausgabeformate hinzugefügt werden können, ohne den bestehenden Code stark ändern zu müssen
    \end{quote}
\end{itemize}
\bigbreak
In einem Projekt ist es wichtig, die verschiedenen Anforderungen nach ihrer Bedeutung zu ordnen, dafür existieren verschiedene Methoden zur Einteilung, diese werden im Folgenden näher erläutert.

Das Kano Modell unterscheidet Anforderungen in Basisfaktoren, Leistungsfaktoren und Begeisterungsfaktoren, um zu zeigen, wie ihre Erfüllung die Zufriedenheit von Nutzern beeinflusst. 
Es macht ersichtlich, welche Merkmale als selbstverständlich erwartet werden, welche direkt die Zufriedenheit steigern und welche einen überraschend positiven Effekt haben \cite{kano1984quality}.

Die RICE Kategorisierung bildet einen nummerischen Score aus Reach × Impact × Confidence / Effort, sodass Anforderungen anhand von Reichweite, Wirkung, Vertrauenswürdigkeit der Schätzung und Aufwand vergleichbar werden. 
Die Methode erlaubt ein quantifiziertes Ranking, das hilft, viele Ideen systematisch zu ordnen \cite{perri2018escaping}.

Die WSJF Kategorisierung priorisiert nach dem Verhältnis Cost of Delay zur geschätzten Dauer, sodass Aufgaben mit dem höchsten wirtschaftlichen Nutzen pro Zeiteinheit zuerst bearbeitet werden. 
Die Methode fokussiert Time to Market und ökonomische Trade offs, indem sie Verzögerungskosten in Relation zur Aufwandsdauer setzt \cite{leffingwell2018safe}.

Die MoSCoW Methode teilt Anforderungen in vier Kategorien ein, Must (Muss), Should (Soll), Could (Kann) und Won’t (wird nicht sein). Muss Anforderungen sind unverzichtbar, ohne sie kann das Projektziel nicht erreicht werden. 
Soll Anforderungen sind ebenfalls wichtig, tragen aber eher zur Verbesserung oder Optimierung bei. Kann Anforderungen sind "nice to have" Funktionen, die umgesetzt werden, wenn Zeit und Ressourcen verfügbar sind. 
Won’t Anforderungen werden für das betrachtete Projekt bewusst ausgeschlossen \cite{agilebusinessconsortium2014dsdm}.
 
Zur Kategorisierung der Folgenden Anforderungen wird die MoSCoW Methode verwendet, diese überzeugt durch ihre einfache Struktur und klare Priorisierung, 
wodurch sie sehr verständlich ist und eine einfache, klare Grundlage für die Entscheidungen schafft.

%. Dafür unterscheidet man zwischen Muss, Soll und Kann Anforderungen. 
%Muss Anforderungen sind unverzichtbar, ohne sie kann das Projektziel nicht erreicht werden. Soll Anforderungen sind ebenfalls wichtig, tragen aber eher zur verbesserung oder Optimierung bei. Kann Anforderungen sind "nice to have" Funktionen,
%die umgesetzt werden, wenn Zeit und Ressourcen verfügbar sind. Zur Priosierung dieser Anforderungen wird häufig die MoSCoW Methode verwendet. Der Begriff steht für Must have, Should have, Could have und Won't have. 
%Diese Methode hilft den Überblick zu behalten und sicherzustellen, dass zuerst die wirklich wichtigen Punkte umgesetzt werden, bevor man sich um zusätzliche Wünsche kümmert.

Die Anforderungen sind in den Kategorien Muss-, Soll- und Kann Anforderungen zugeordnet. Eine Übersicht bietet die folgende Tabelle.

\noindent
\begin{longtable}{|p{3cm}|p{\dimexpr\textwidth-3cm-2\tabcolsep-2\arrayrulewidth\relax}|}
\hline
\textbf{Kategorie} & \textbf{Beschreibung der Anforderung} \\
\hline
\endhead
\hline
\endfoot
Muss & Korrekter Datenimport der eingegebenen Daten \\
\hline
Muss & Zuverlässigkeit der Korrektheit der Testfälle \\
\hline
Soll & Möglichkeit zur Erweiterbarkeit neuer Datenformate \\
\hline
Soll & Automatische Erzeugung von Testfällen basierend auf den eingelesenen Daten \\
\hline
Soll & Effiziente Verarbeitung großer Datenmengen \\
\hline
Soll & Wartbarkeit des Generators \\
\hline
Kann & Unterstützung mehrerer Ausgabeformate \\
\hline
Kann & Einfache Bedienung für die Anwender \\
\hline
\end{longtable}


%Zu den \textbf{Muss Anforderungen} zählt der korrekte Import der eingegebenen Daten. 
%Diese Funktion ist wesentlich, da ohne einen fehlerfreien und vollständigen Datenimport keine Testfälle erstellt werden können. 
%Ebenso unverzichtbar ist die Gewährleistung der Korrektheit der Testfälle, da nur fehlerfreie Ergebnisse eine zuverlässige Überprüfung der Logik in der Steuererklärungsoftware ermöglichen.

Zu den \textbf{Muss Anforderungen} zählt der korrekte Import der eingegebenen Daten. Diese Funktion ist wesentlich, da ohne einen fehlerfreien und vollständigen Datenimport keine Testfälle erstellt werden können. 
Ebenso unverzichtbar ist die Gewährleistung der Korrektheit der Testfälle, da nur fehlerfreie Ergebnisse eine zuverlässige Überprüfung der Logik in der Steuererklärung Software ermöglichen. 
MoSCoW klassifiziert diese Anforderungen als Must da ohne diese der Testfallgenerator seine Funktion nicht erfüllen kann.

%Zu den \textbf{Soll Anforderungen} gehört die Möglichkeit, neue Datenformate zu unterstützen, um die Software flexibel an zukünftige Anforderungen anpassen zu können, da diese Funktion für zukünftige Erweiterungen wichtig ist und für den Anfang nicht relevant ist.  
%Auch die automatische Erzeugung von Testfällen auf Basis der eingelesenen Daten ist ein wesentlicher Bestandteil, der den Prozess deutlich beschleunigt, im Bedarfsfall jedoch auch manuell umgesetzt werden könnte. 
%Die effiziente Verarbeitung großer Datenmengen trägt zu einer hohen Leistungsfähigkeit bei, ist jedoch nicht zwingend für die Grundfunktion erforderlich. 
%Die Wartbarkeit des Generators ist ebenfalls von Bedeutung, um spätere Anpassungen und Fehlerbehebungen mit geringem Aufwand vornehmen zu können.
%Die genannten Anforderungen fallen in das Sollkriterium, da der Testfallgenerator auch ohne diese Funktion einsatzfähig ist und nicht zwingend darauf angewiesen ist.

Zu den \textbf{Soll Anforderungen} gehört die Möglichkeit, neue Datenformate zu unterstützen, um die Software flexibel an zukünftige Anforderungen anpassen zu können, da diese Funktion für zukünftige Erweiterungen wichtig ist und für den Anfang nicht relevant ist. 
Auch die automatische Erzeugung von Testfällen auf Basis der eingelesenen Daten ist ein wesentlicher Bestandteil, der den Prozess deutlich beschleunigt, im Bedarfsfall jedoch auch manuell umgesetzt werden könnte. 
Die effiziente Verarbeitung großer Datenmengen trägt zu einer hohen Leistungsfähigkeit bei, ist jedoch nicht zwingend für die Grundfunktion erforderlich. 
Die Wartbarkeit des Generators ist ebenfalls von Bedeutung, um spätere Anpassungen und Fehlerbehebungen mit geringem Aufwand vornehmen zu können. 
Die MoSCoW Methode ordnet diese Anforderungen als Should ein, da sie die Qualität und Erweiterbarkeit des Testfallgenerators deutlich erhöhen aber nicht zwingend für die Grundfunktionalität erforderlich sind.

%Zu den \textbf{Kann Anforderungen} zählen die Unterstützung mehrerer Ausgabeformate, die den Anwendern zusätzliche Möglichkeiten bei der Weiterverarbeitung der Testergebnisse bietet, 
%sowie eine besonders einfache Bedienung, die den Einstieg erleichtert und die Nutzererfahrung verbessert. 
%Diese Funktionen sind nicht zwingend erforderlich, damit die Software ihre funktionalität erfüllt, sondern um den Komfort und die Flexibilität der Nutzer zu erhöhen. 
%Daher sind diese als Kann Anforderungen kategorisiert, da der Testfallgenerator ohne diese Erweiterungen einsatzfähig bleibt.

Zu den \textbf{Kann Anforderungen} zählen die Unterstützung mehrerer Ausgabeformate, die den Anwendern zusätzliche Möglichkeiten bei der Weiterverarbeitung der Testergebnisse bietet, sowie eine besonders einfache Bedienung, die den Einstieg erleichtert und die Nutzererfahrung verbessert. 
MoSCoW klassifiziert diese Anforderungen als Could, da sie den Bedienkomfort und die Flexibilität des Testfallgenerators erhöhen, für die erste funktionsfähige Version des Testfallgenerators jedoch nicht zwingend erforderlich sind.

Damit ist die Priorisierung abgeschlossen und die Anforderungen sind klar nach Wichtigkeit geordnet. Die Einordnung bildet die Grundlage für die Planung und zeigt, welche Funktionen zuerst umgesetzt werden müssen und welche bei Bedarf verschoben werden könne.
Prioritäten sollten regelmäßig übergeprüft und bei neuen Erkenntnissen oder geänderten Rahmenbedingungen angepasst werden.

%\section{Lösungsansatz zur Umsetzung der Anforderungen}
%
%Um die gestellten Anforderungen effizient umzusetzen, wurde der Testfallgenerator in der Programmiersprache Java entwickelt. 
%Das grundlegende Konzept besteht darin, unterschiedliche Eingabedatenformate, wie zum Beispiel \ac{xml} oder Excel Dateien, einzulesen und diese anschließend intern weiterzuverarbeiten. 
%Auf Basis dieser verarbeiteten Daten werden automatisch Testfälle generiert.
%
%Zuerst werden die eingelesenen Daten in Java Objekte umgewandelt, damit sie im Programm übersichtlich und gut organisiert verarbeitet werden können. 
%Danach werden die Testfälle nach festgelegten Regeln erstellt, sodass die Ergebnisse zuverlässig und nachvollziehbar sind. 
%Zum Schluss werden die fertigen Testfälle in einem passenden Format ausgegeben, das für die weiteren Schritte oder die Testumgebung verwendet werden kann.
%
%Besonders wichtig bei der Entwicklung des Testfallgenerators war der Aufbau des Codes. 
%Dieser wurde bewusst so gestaltet, dass er in Zukunft leicht erweitert und an neue Anforderungen angepasst werden kann. 
%Das ermöglicht zum Beispiel die einfache Integration neuer Datenformate oder die Anpassung an Änderungen in der Datenstruktur, ohne dass bestehende Funktionen beeinträchtigt werden.
%
%Ein zentrales Ziel der Lösung war es, eine einfache und zuverlässige Funktion sicherzustellen, die gleichzeitig flexibel genug ist, um auch künftigen Anforderungen gerecht zu werden. 
%So kann der Testfallgenerator langfristig und effizient in verschiedenen Anwendungsszenarien eingesetzt werden.
%	Umsetzung
%\chapter{Konzeption} \label{kap: Konzeption}

\lstdefinelanguage{XML}{
  morestring=[b]",
  morecomment=[s]{<!--}{-->},
  stringstyle=\color{red},
  identifierstyle=\color{blue},
  morekeywords={xmlns,version,type}
}

\lstset{
  language=XML,
  basicstyle=\ttfamily\tiny,    % kleine Schrift
  numbers=left,
  numberstyle=\tiny,
  keywordstyle=\color{blue},
  commentstyle=\color{gray},
  stringstyle=\color{red},
  breaklines=true,
  frame=single,
  captionpos=b
}



\chapter{Implementierung und Laufzeit} \label{chap:Implementierung und Laufzeit}
%Dieser Abschnitt beschreibt das Einlesen und Verarbeiten von \ac{xml} und Excel Daten, definiert die Logik zur Generierung von Testfällen und begründet die Wahl der genutzten Werkzeuge. 
%Dieser Abschnitt beschreibt das Einlesen und Verarbeiten von XML und Excel Dateien, die Logik zur Testfallgenerierung, die Wahl der Werkzeuge, die Funktionstests mit JUnit sowie beispielhafte Testfallausgaben.
Dieser Abschnitt beschreibt das Einlesen und Verarbeiten von XML und Excel Dateien, die Logik zur Testfallgenerierung, die Wahl der Werkzeuge sowie beispielhafte Testfallausgaben und beschreibt die Laufzeit.

\section{Lösungsansatz zur Umsetzung der Anforderungen}

Um die gestellten Anforderungen effizient umzusetzen, wurde der Testfallgenerator in der Programmiersprache Java entwickelt. 
Das grundlegende Konzept besteht darin, unterschiedliche Eingabedatenformate, wie zum Beispiel \ac{xml} oder Excel Dateien, einzulesen und diese anschließend intern weiterzuverarbeiten. 
Auf Basis dieser verarbeiteten Daten werden automatisch Testfälle generiert.

Zuerst werden die eingelesenen Daten in Java Objekte umgewandelt, damit sie im Programm übersichtlich und gut organisiert verarbeitet werden können. 
Danach werden die Testfälle nach festgelegten Regeln erstellt, sodass die Ergebnisse zuverlässig und nachvollziehbar sind. 
Zum Schluss werden die fertigen Testfälle in einem passenden Format ausgegeben, das für die weiteren Schritte oder die Testumgebung verwendet werden kann.

%Besonders wichtig bei der Entwicklung des Testfallgenerators war der Aufbau des Codes. 
%Dieser wurde bewusst so gestaltet, dass er in Zukunft leicht erweitert und an neue Anforderungen angepasst werden kann. 
%Das ermöglicht zum Beispiel die einfache Integration neuer Datenformate oder die Anpassung an Änderungen in der Datenstruktur, ohne dass bestehende Funktionen beeinträchtigt werden.

Besonders wichtig bei der Entwicklung des Testfallgenerators war der strukturierte Aufbau des Codes. 
Dieser wurde bewusst so gestaltet, dass er in Zukunft leicht erweitert und an neue Anforderungen angepasst werden kann.
Zur Sicherstellung der Erweiterbarkeit basiert die Architektur auf klar definierten Reader und Writer Interfaces, sodass neue Eingabe und Ausgabeformate (z. B. CSV oder Datenbanken) als separate Implementierungen ergänzt werden können, 
ohne die Kernlogik des Testfallgenerators tiefgreifend zu verändern. 
Dadurch wird die Integration weiterer Formate oder die Anpassung an geänderte Datenstrukturen möglich, ohne bestehende Funktionen zu beeinträchtigen.

Ein zentrales Ziel der Lösung war es, eine einfache und zuverlässige Funktion sicherzustellen, die gleichzeitig flexibel genug ist, um auch künftigen Anforderungen gerecht zu werden. 
So kann der Testfallgenerator langfristig und effizient in verschiedenen Anwendungsszenarien eingesetzt werden.

\section{Implementierung}
In diesen Abschnitt, wird die Implementierung des Testfallgenerators genauer definiert.

\subsection{Einlesen und Verarbeiten von XML/Excel Dateien}
Der Einlese und Verarbeitungsprozess von \ac{xml} und Excel Dateien bildet die Grundlage für die Arbeit des Testfallgenerators.
Beide Formate unterscheiden sich in ihrer Struktur, \ac{xml} nutzt eine hierarische Baumdarstellung, während Excel tabellarisch aufgebaut ist.
Im ersten Schritt werden die Dateien geöffnet und die relevanten Daten wie laufende Nummer, 
Eingabewerte und optionale Beschreibungen ausgelesen. 

Bei \ac{xml} erfolgt dies über die Navigation durch die Knoten, 
dazu erstellt man zunächst eine \textit{DocumentBuilderFactory}, die den Parser konfiguriert. 
Mit einem \textit{DocumentBuilder} wird die \ac{xml} Datei eingelesen und ein \ac{dom} Baum im Speicher aufgebaut, der das gesamte Dokument abbildet.
Über Methoden wie getDocumentElement(), lässt sich das Root Element auswählen, während getElementByTagName() gezielt bestimmte Elemente anspricht.
Attribute können mit getAttribute() ausgelesen werden, und die Inhalte der Elemente lassen sich über ihre Unterelemente verarbeiten.

In Excel werden Daten in Tabellenform gespeichert, und bestehen aus mehreren Tabellenblättern die Sheet genannt werden, die jeweils Zeilen und Zellen enthalten. 
Ein Workbook stellt die ganze Datei dar und kann mit WorkbookFactory.create() geöffnet werden. Um die Daten auszulesen, 
durchläuft das Programm die einzelnden Tabellenblätter und liest die Daten aus den Zeilen und Zellen nacheinander.
Dabei ist es wichtig die verschiedenen Zelltypen zu beachten, wie Text STRING, Zahlen NUMERIC oder Wahrheitswerte BOOLEAN. 
Apache POI bietet Methoden wie zum Beispiel getStringCellValue() um die Daten aus den einzelnden Zellen auszulesen.
Beim auslesen von \ac{xml} als auch Excel Dateien ist es wichtig auf die Fehlerbehandlung zu achten zum Beispiel durch die Nutzung von try-catch Anweisungen.
Das Ziel ist es die gleichen Daten zu erfassen unabhängig vom Format.

Im nächsten Schritt werden die Daten in eine gemeinsame interne Struktur gebracht. 
Die eingelesenen Daten werden zur Laufzeit in Java Objekten gespeichert, die im Arbeitsspeicher (Heap) der Java Virtual Machine abgelegt sind, diese Java Objekte werden \ac{dto} genannt. 
Diese sind einfache Java Klassen, die Daten strukturiert und ohne Logik kapseln. 
Erweitert wird der Abschnitt durch eine Beschreibung des internen Datenmodells und der Architektur. 
Eingelesene Werte werden in Java Klassen vom Typ Feld abgelegt, die Klasse Feld speichert die E Nummer\footnotemark[1] und den dazugehörigen Eingabewert. 
Mehrere Feldobjekte bilden ein Formular, zum Beispiel GW1\footnotemark[1], GW2\footnotemark[1] oder GW4\footnotemark[1], außerdem bilden mehrere Formulare zusammen eine Erklärung, diese enthält die Daten der Steuererklärung. 
Zur klaren Trennung der Verantwortlichkeiten existieren ein gemeinsames Reader Interface und ein gemeinsames Writer Interface. 
Konkrete Implementierungen XMLReader, ExcelReader, XMLWriter und ExcelWriter implementieren diese Schnittstellen und übernehmen jeweils das formatspezifische Parsen. 
Der Generator setzt die DTOs zu Testfällen zusammen und Writer geben die Testfälle in den gewünschten Formaten aus. 
%Sie ermöglichen den effizienten und sicheren Datenaustausch zwischen verschiedenen Systemkomponenten oder Schichten. 
%Beispielsweise können Methoden direkt auf eingelesene Daten aus Excel oder \ac{xml} Dateien zugreifen.
%Dadurch kann der Testfallgenerator unabhängig vom Ursprungsformat immer mit einer einheitlichen internen Datenstruktur arbeiten. 
\footnotetext[1]{Die Begriffe \textquotedbl{}E Nummer\textquotedbl{}, \textquotedbl{}GW1\textquotedbl{}, \textquotedbl{}GW2\textquotedbl{} und \textquotedbl{}GW3\textquotedbl{} werden im Kapitel \ref{name: Beispielhafte Testfallausgaben} weiter Erläutert}

\subsection{Ablauf zur Generierung der Testfälle}
Bei der Generierung von Testfällen im Steuerumfeld geht es darum, aus vorhandenen Daten systematisch Testfälle zu erstellen.
Die Daten sind in \ac{xml} oder Excel Dateien vorhanden und enthalten zum Beispiel die \ac{steuerid}, Einkommenswerte oder den Familienstand.
Damit die Testfälle einheitlich aufgebaut sind, bekommen sie eine feste Struktur mit einer laufenden Nummer, den Eingabewerten und dem erwarteten Ergebnis.
Auf dieser Basis erzeugt der Testfallgenerator die Testfälle. 
%Neben den normalen Fällen aus den Daten werden auch Sonderfälle gebildet, bei denen zum Beispiel die \ac{steuerid} fehlt, falsche Einkommenswerte auftreten oder nicht nachvollziehbare Eingaben getätigt werden. 
%So werden nicht nur Standardfälle, sondern auch mögliche Fehler oder Ausnahmen berücksichtigt.
Am Ende werden alle Testfälle geprüft, durchnummeriert und gespeichert. Dadurch entsteht eine einheitliche Sammlung, die für den Testverlauf genutzt werden kann.


\subsection{Wahl und Begründung der Werkzeuge}
%Für die Verarbeitung der Daten wurde die Programmiersprache Java und passende Bibliotheken wie \ac{jaxb}, ApachePOI und JUnit verwendet.
%Für die Verarbeitung von \ac{xml} Dateien wurde die \ac{dom} API verwendet, weil sie das \ac{xml} Dokument als Baumstruktur im Arbeitsspeicher abbildet. 
%Dadurch können die einzelnen Elemente leicht zugänglich und bearbeitbar gemacht werden.
%Einzelne Elemente und Attribute lassen sich direkt auslesen, da die \ac{dom} API ein Teil von Java ist, dadurch sind keine externen Bibliotheken notwendig.
%Für Excel Dateien wird Apache POI verwendet. Die Bibliothek unterstützt viele Methoden um Tabellenblätter, Zeilen und Zellen auszulesen. 
%Apache POI ist in Java Projekten weit verbreitet, wodurch die Integration erleichtert wird.
%
%Die Auswahl dieser Werkzeuge ermöglicht eine flexible und wartbare Lösung.
%Daten aus beiden Formaten können ausgelesen und in java Objekte überführt werden und für die Testfallgenerierung genutzt werden.

Für die Verarbeitung der Daten wurde die Programmiersprache Java gewählt. 
Java bietet eine plattformunabhängige Umgebung sowie viele verfügbare Bibliotheken, die die Entwicklung erleichtern.

Zur Verarbeitung von \ac{xml} Dateien kam die \ac{dom} API zum Einsatz. 
Die \ac{dom} API bildet das \ac{xml} Dokument als Baumstruktur im Arbeitsspeicher ab, wodurch einzelne Elemente und Attribute einfach zugänglich und veränderbar sind. 
Dies ist besonders hilfreich, wenn umfangreiche Manipulationen oder mehrfache Zugriffe auf verschiedene Teile der \ac{xml} Struktur nötig sind. 
Alternativ wären SAX oder StAX möglich gewesen, die eher eventbasiert arbeiten und weniger Speicher benötigen. Allerdings sind diese APIs eher für sequentielle Lesevorgänge geeignet und weniger intuitiv bei komplexeren Bearbeitungen. 
Daher wurde die \ac{dom} API bevorzugt, da sie eine einfachere und übersichtlichere Handhabung für diesen Anwendungsfall bietet. 
Zudem ist die \ac{dom} API direkt in Java integriert, sodass keine externen Abhängigkeiten erforderlich sind.
Zusätzlich ermöglicht es eine flexible Bearbeitung der \ac{xml} Daten und erleichtert spätere Anpassungen oder Erweiterungen der \ac{xml} Struktur.

Für die Verarbeitung von Excel Dateien wurde die Bibliothek Apache POI verwendet. 
Apache POI unterstützt umfassend das Lesen und Schreiben von Excel Dateien in verschiedenen Formaten (.xls und .xlsx). 
Die Bibliothek bietet eine Vielzahl von Methoden, um Tabellenblätter, Zeilen und Zellen zu lesen oder zu bearbeiten. 
Im Vergleich zu Alternativen wie JExcelAPI oder OpenCSV ist Apache POI aktueller, unterstützt mehr Exce Formate und wird von einer großen Entwicklergemeinschaft gepflegt. 
Dadurch ist die Integration in Java Projekte einfach und die langfristige Wartbarkeit gesichert.

%Zur Absicherung der Implementierung wurde JUnit als Testframework eingesetzt, da dies in der Java Umgebung weit verbreitet ist und eine einfache und  Möglichkeit bietet, automatisierte Tests zu schreiben und auszuführen.
%Zur Absicherung der Implementierung wurde JUnit als Testframework eingesetzt. Durch seine Verbreitung in der Java Umgebung bietet es eine einfache und zugleich solide Möglichkeit, automatisierte Tests zu entwickeln und auszuführen.


Bibliotheken mit guter Dokumentation und aktiver Entwicklergemeinschaft unterstützen die Wartbarkeit, da Fehler schnell behoben und Anpassungen leicht umgesetzt werden können.
Durch den modularen Aufbau und der klaren Trennung zwischen Einlesen, Datenmodellierung und Testfallgenerierung ergibt sich eine flexible Lösung, wodurch Erweiterungen ohne großen Aufwand möglich sind.
Durch die Kombination dieser Technologien entstand eine leistungsfähige, flexible und wartbare Lösung, die Daten aus beiden Formaten zuverlässig auslesen, in Java Objekte überführen und für die Testfallgenerierung verwenden kann.

%\section{Tests}
%In folgenden Abschnitt werden JUnit Tests und die Überprüfung der Testfalldaten genauer beschrieben.
%%\chapter{Tests} \label{chap:Tests}
%%In diesem Kapitel werden die Funktionstests des Generators mithilfe von JUnit vorgestellt. 
%%Dabei wird insbesondere auf die beispielhaften Testfallausgaben sowie auf die Korrektheit und Vollständigkeit der generierten Testfälle eingegangen.
%
%\subsection{Funktionstests des Generators (JUnit Tests)}
%Um die Funktionsweise des Generators zu überprüfen, wurden JUnit-Tests eingesetzt.
%Mit diesen Tests kann kontrolliert werden ob die Eingabedaten aus \ac{xml} und Excel Dateien korrekt eingelesen und in die vorgesehene interne Struktur übertragen werden. 
%Zusätzlich wurde getestet, 
%ob die erzeugten Testfälle dem festgelegten Schema entsprechen und die Ausgaben vollständig und fehlerfrei sind. 
%Die Tests berücksichtigen verschiedene Szenarien, wie beispielsweise das Einlesen unterschiedlicher \ac{xml} Elemente mit variierenden Attributen sowie Excel Dateien mit diversen Zellinhalten. 
%Die Testfälle sind so aufgebaut, dass sie die Vollständigkeit und Korrektheit der eingelesenen Daten sicherstellen. 
%Hierbei wird geprüft, ob alle Werte korrekt den jeweiligen internen Objekten zugeordnet werden. 
%Die automatisierte Ausführung der Tests ermöglicht eine schnelle und wiederholbare Überprüfung der Funktionsfähigkeit des Generators, sodass nach Änderungen am Code zeitnah die Zuverlässigkeit des Systems bestätigt werden kann.

\subsection{Beispielhafte Testfallausgaben}\label{name: Beispielhafte Testfallausgaben}
%Um die Arbeitsweise des Testfallgenerators besser verstehen zu können, wird im Folgenden der Aufbau der Grundsteuererklärung beschrieben.
%Die Grundsteuererklärung ist in 3 Formulare unterteilt: GW1, GW2, GW4.
Zur besseren Verständlichkeit der Funktionsweise des Testfallgenerators wird im Folgenden der Aufbau der Grundsteuererklärung erläutert. 
Diese gliedert sich in drei Formulare: GW1, GW2 und GW4.

%\footnote{Im Folgenden wird die Bezeichnung "Feldkennnummer\textquotedbl{} verwendet, diese sind eindeutige Kennnummern die jedem Feld in einem Formular zugewiesen sind. Sie besitzen eine interne Bedeutung und werden zur internen Verarbeitung der Feldinhalte verwendet.} und einem Eingabewert.


Das Formular GW1 dient als Hauptvordruck für die Grundsteuererklärung. 
Es erfasst grundlegende Informationen zur wirtschaftlichen Einheit, einschließlich Angaben zum Eigentümer, zur Lage des Grundstücks, zum Aktenzeichen sowie zum Grund der Feststellung. 
Dieses Formular bildet die Basis für die weiteren Angaben und ist für alle Grundstücke erforderlich \cite{elster_bw2025}.

Die Anlage GW2 enthält spezifische Daten zum Grundstück selbst. Hier werden Informationen wie die Gemarkung, die Flurstücksnummer, die Grundstücksfläche, der Bodenrichtwert sowie die Nutzung des Grundstücks abgefragt. 
Bei Miteigentum sind zudem die jeweiligen Eigentumsanteile anzugeben. Dieses Formular ist notwendig, wenn es sich nicht um ein land- und forstwirtschaftliches Grundstück handelt \cite{elster_bw2025}.

Das Formular GW4 wird verwendet, wenn das Grundstück ganz oder teilweise von der Grundsteuer befreit oder eine Ermäßigung der Steuermesszahl in Anspruch genommen wird. 
Hier sind die Art der Befreiung oder Vergünstigung sowie die entsprechenden gesetzlichen Grundlagen anzugeben. 
Dieses Formular ist insbesondere relevant für Grundstücke, die beispielsweise gemeinnützigen Zwecken dienen oder unter Denkmalschutz stehen \cite{elster_bw2025}.

Die eingegebenen Daten in der Grundsteuererklärung werden elektronisch gespeichert. 
Dabei sind die einzelnen Eingabefelder jeweils mit sogenannten E Nummern versehen. 
Diese E Nummern dienen der internen Verarbeitung und Zuordnung der Daten durch die Finanzbehörden.

E Nummern sind interne Kennzeichnungen für Eingabefelder in den elektronischen Grundsteuerformularen, die über das ELSTER Portal eingereicht werden. 
Sie erleichtern den Finanzämtern die strukturierte Verarbeitung und Zuordnung der Daten.
Jede E Nummer wird mit einem E und einer 7 stelligen Zahl definiert zum Beispiel \textquotedbl{}E1234567\textquotedbl{}.
Informationsfelder wie Laufendenummer, Beschreibung oder ähnliche besitzen keine Feldkennnummer und bekommen ihren Namen als eindeutige Feldkennung.
Die Feldkennnummer wird innerhalb der Datei mit \textquotedbl{}nr\textquotedbl{} und der Eingabewert mit \textquotedbl{}wert\textquotedbl{} abgekürzt.
Für Bürgerinnen und Bürger sind diese Nummern meist unsichtbar und richten sich hauptsächlich an Softwareentwickler, Steuerberater und Finanzbeamte \cite{fm_bw_elnr2025}.


eine beispielhafte Datei der erzeugten Testfälle gezeigt. 
Diese verdeutlicht den Aufbau der Testfälle. 
Jeder Testfall hat dabei eine eindeutige laufende Nummer, eine kurze Beschreibung und die dazugehörigen Eingabedaten.

Zur Veranschaulichung beispielhafter Testfallausgaben wurde \ac{xml} verwendet, um den Aufbau und die Struktur darzustellen. Hierfür wurde das Formular GW2 als Teil einer Erklärung am folgenden Beispiel gezeigt. \newline

%<GW1>
%<Feld nr="lfdNr" wert="4"/>
%<Feld nr="Bezeichnung" wert="Sachbearbeitung"/>
%<Feld nr="Kurzbeschreibung" wert="Bauerwartungsland"/>
%<Feld nr="Stichtag" wert="Sat Jan 01 00:00:00 CET 2022"/>
%<Feld nr="Aktenzeichen" wert="083500120680040016"/>
%<Feld nr="Finanzamt" wert="Karlsruhe-Stadt"/>
%<Feld nr="E7401124" wert="Adalbert-Stifter-Straße"/>
%<Feld nr="E7401125" wert="9"/>
%<Feld nr="E7401131" wert="Wohnungseigentum 4"/>
%<Feld nr="E7401121" wert="76199"/>
%<Feld nr="E7401122" wert="Karlsruhe"/>
%<Feld nr="E7401141" wert="Gaggenau"/>
%<Feld nr="E7411702" wert="Bei Rückfragen bitte telefonisch am Vormittag 0761123456"/>
%<Feld nr="lfdNr2" wert="001"/>
%<Feld nr="E7404518" wert="14.03.1967"/>
%<Feld nr="E7404513" wert="Tom"/>
%<Feld nr="E7404511" wert="Maus"/>
%<Feld nr="E7404524" wert="Im Efeu"/>
%<Feld nr="E7404571" wert="2"/>
%<Feld nr="lfdNr3" wert="002"/>
%<Feld nr="E74045182" wert="15.04.1967"/>
%<Feld nr="E74045132" wert="Jerry"/>
%<Feld nr="E74045112" wert="Maus "/>
%<Feld nr="E74045242" wert="Im Näpfle"/>
%<Feld nr="E74045252" wert="5"/>
%<Feld nr="E74045402" wert="74078"/>
%<Feld nr="E74045222" wert="Heilbronn"/>
%</GW1>
%<GW4/>

\begin{lstlisting}[caption={Beispielhafte Testfallausgaben in XML}] 
<DatenTeil>
<Erklaerung LfdNr="4">
<GW2>
<Feld nr="E7421322" wert="1"/>
<Feld nr="E7403010" wert="71"/>
<Feld nr="E7403011" wert="730"/>
</GW2>
</Erklaerung>
</DatenTeil>
\end{lstlisting}


%\section{Korrektheit und Vollständigkeit der Testfälle}
%Bei der Arbeit mit Testfällen ist es besonders wichtig, dass alle Angaben korrekt und vollständig sind. 
%Korrektheit bedeutet, dass die eingetragenen Werte genau den Daten aus der Quelle entsprechen und keine Fehler enthalten.
%So muss zum Beispiel die laufende Nummer korrekt übernommen werden um korrekt wiedergegeben werden zu können.
%Die Vollständigkeit bezieht sich darauf, das alle Pflichtfelder eines Testfalls ausgefüllt sind. 
%Fehlende oder unvollständige Angaben können dazu führen, dass Testfälle nicht korrekt ausgeführt werden oder im Testprozess Fehler auftreten.
%Ein weiterer wichtiger Punkt ist, dass die Testfälle klar und übersichtlich aufgebaut sind. Nur so können die Testfälle bei Bedarf problemlos kontrolliert werden.
%Eine einheitliche Struktur  hilft nicht nur den Nutzern sondern auch dem Generator, 
%da so Daten ohne schwierigkeiten verarbeitet werden können.

\section{Laufzeit}
In diesem Abschnitt wird die Laufzeit der Konvertierung von einer Excel Datei in eine XML Datei des Testfallgenerators untersucht.

Das System, auf dem die Messungen durchgeführt wurden, besteht aus einem Intel(R) Core(TM) i7 10750H (2,60 GHz), 32 GB RAM, einer NVIDIA Quadro T1000 (4 GB) sowie integrierter Intel UHD Graphics (128 MB). 
Das Betriebssystem leuft unter Windows 10 Enterprise und verwendet zwei SSDs (Toshiba KXG6AZNV512G 477 GB, Samsung MZVLB1T0HBLR 000L7 954 GB), wobei die Messläufe lokal auf einer SSD ausgeführt wurden.

Für jede Dateigröße (50 KB, 100 KB, 150 KB, 200 KB, 250 KB) wurde die Konvertierung von Excel nach XML fünfmal ausgeführt. 
Der arithmetische Mittelwert dieser Daten bildet die Messgröße der Gesamtverarbeitungszeit vom Programmstart bis zur fertigen XML Datei pro Größe. 
Die Läufe wurden lokal auf einer SSD unter ruhenden Systembedingungen durchgeführt.

%\textbf{Ergebnisse der Messwerte}
%\begin{itemize}
%  \item 50KB in 937 ms \rightarrow  \approx 50.36 KB/s
%  \item 100KB in 1167 ms
%  \item 150KB in 1456 ms
%  \item 200KB in 1771 ms
%  \item 250KB in 1985 ms
%\end{itemize}

\textbf{Ergebnisse der Messwerte}
\begin{itemize}
  \item 50\,KB in 937\,ms $\rightarrow\ \approx 50{.}36\ \mathrm{KB/s}$
  \item 100\,KB in 1167\,ms $\rightarrow\ \approx 85{.}69\ \mathrm{KB/s}$
  \item 150\,KB in 1456\,ms $\rightarrow\ \approx 103{.}05\ \mathrm{KB/s}$
  \item 200\,KB in 1771\,ms $\rightarrow\ \approx 112{.}87\ \mathrm{KB/s}$
  \item 250\,KB in 1985\,ms $\rightarrow\ \approx 125{.}94\ \mathrm{KB/s}$
\end{itemize}

Die Messgrößen zeigen, dass die Laufzeit mit der Dateigröße linear wächst und sich durch die Formel T(n) = k·n + c beschreiben lässt, wobei n die Dateigröße, c ein fester Startaufwand und k die Zeit pro Daten Einheit bezeichnet. 
Damit ist die asymptotische Laufzeit linear, also $T(n) \in $ O(n).

Die folgende Grafik zeigt die Messwerte und veranschaulicht das Laufzeitverhalten.

\begin{figure}[H]
    \centering
    \includegraphics[width=0.5\textwidth]{./img/Laufzeit Excel zu XML.png}
    \caption{Laufzeit Analyse Excel zu XML}
    \label{fig:bild2}
\end{figure}
%\chapter{Entwicklung des Prototyps anhand der Konzeption} \label{kap:Umsetzung}

%\lstdefinelanguage{XML}{
  morestring=[b]",
  morecomment=[s]{<!--}{-->},
  stringstyle=\color{red},
  identifierstyle=\color{blue},
  morekeywords={xmlns,version,type}
}

\lstset{
  language=XML,
  basicstyle=\ttfamily\tiny,    % kleine Schrift
  numbers=left,
  numberstyle=\tiny,
  keywordstyle=\color{blue},
  commentstyle=\color{gray},
  stringstyle=\color{red},
  breaklines=true,
  frame=single,
  captionpos=b
}

\chapter{Tests} \label{chap:Tests}
In diesem Kapitel werden die Funktionstests des Generators mithilfe von JUnit vorgestellt. 
Dabei wird insbesondere auf die beispielhaften Testfallausgaben sowie auf die Korrektheit und Vollständigkeit der generierten Testfälle eingegangen.

\section{Funktionstests des Generators (JUnit Tests)}
Um die Funktionsweise des Generators zu überprüfen, wurden JUnit-Tests eingesetzt.
Mit diesen Tests kann kontrolliert werden ob die Eingabedaten aus \ac{xml} und Excel Dateien korrekt eingelesen und in die vorgesehene interne Struktur übertragen werden. 
Zusätzlich wurde getestet, 
ob die erzeugten Testfälle dem festgelegten Schema entsprechen und die Ausgaben vollständig und fehlerfrei sind. 
Die Tests berücksichtigen verschiedene Szenarien, wie beispielsweise das Einlesen unterschiedlicher \ac{xml} Elemente mit variierenden Attributen sowie Excel Dateien mit diversen Zellinhalten. 
Die Testfälle sind so aufgebaut, dass sie die Vollständigkeit und Korrektheit der eingelesenen Daten sicherstellen. 
Hierbei wird geprüft, ob alle Werte korrekt den jeweiligen internen Objekten zugeordnet werden. 
Die automatisierte Ausführung der Tests ermöglicht eine schnelle und wiederholbare Überprüfung der Funktionsfähigkeit des Generators, sodass nach Änderungen am Code zeitnah die Zuverlässigkeit des Systems bestätigt werden kann.

\section{Beispielhafte Testfallausgaben} \label{name: Beispielhafte Testfallausgaben}
Um die Arbeitsweise des Generators besser verstehen zu können, werden im Folgenden einige Beispiele der erzeugten Testfälle gezeigt. 
Diese Beispiele machen deutlich, wie die Testfälle aufgebaut sind und welche Informationen sie enthalten. 
Jeder Testfall hat dabei eine eindeutige laufende Nummer, eine kurze Beschreibung und die dazugehörigen Eingabedaten.
Die eingelesenen Informationen sind im DatenTeil zu finden, GW1, GW2 und GW4 stehen für die einzelnen Formulare innerhalb der Erklärung. 
Die Feldkennnummer wird mit 'nr' abgekürzt und Felder ohne Kennnummer bekommen ihre Bezeichung als Feldkennnummer.

Zur Veranschaulichung beispielhafter Testfallausgaben wurde \ac{xml} verwendet, um den Aufbau und die Struktur darzustellen. \newline

\begin{lstlisting}[caption={Beispielhafte Testfallausgaben in XML}] 
<Elster xmlns="http://www.elster.de/elsterxml/schema/v11">
<Verfahren>ElsterErklaerung</Verfahren>
<DatenArt>GrundSteuerBW</DatenArt>
<DatenTeil>
<Erklaerung LfdNr="4">
<GW1>
<Feld nr="lfdNr" wert="4"/>
<Feld nr="Bezeichnung" wert="Sachbearbeitung"/>
<Feld nr="Kurzbeschreibung" wert="Bauerwartungsland"/>
<Feld nr="Stichtag" wert="Sat Jan 01 00:00:00 CET 2022"/>
<Feld nr="Aktenzeichen" wert="083500120680040016"/>
<Feld nr="Finanzamt" wert="Karlsruhe-Stadt"/>
<Feld nr="E7401124" wert="Adalbert-Stifter-Straße"/>
<Feld nr="E7401125" wert="9"/>
<Feld nr="E7401131" wert="Wohnungseigentum 4"/>
<Feld nr="E7401121" wert="76199"/>
<Feld nr="E7401122" wert="Karlsruhe"/>
<Feld nr="E7401141" wert="Gaggenau"/>
<Feld nr="E7411702" wert="Bei Rückfragen bitte telefonisch am Vormittag 0761123456"/>
<Feld nr="lfdNr2" wert="001"/>
<Feld nr="E7404518" wert="14.03.1967"/>
<Feld nr="E7404513" wert="Tom"/>
<Feld nr="E7404511" wert="Maus"/>
<Feld nr="E7404524" wert="Im Efeu"/>
<Feld nr="E7404571" wert="2"/>
<Feld nr="lfdNr3" wert="002"/>
<Feld nr="E74045182" wert="15.04.1967"/>
<Feld nr="E74045132" wert="Jerry"/>
<Feld nr="E74045112" wert="Maus "/>
<Feld nr="E74045242" wert="Im Näpfle"/>
<Feld nr="E74045252" wert="5"/>
<Feld nr="E74045402" wert="74078"/>
<Feld nr="E74045222" wert="Heilbronn"/>
</GW1>
<GW2>
<Feld nr="E7421322" wert="1"/>
<Feld nr="E7403010" wert="71"/>
<Feld nr="E7403011" wert="730"/>
</GW2>
<GW4/>
</Erklaerung>
</DatenTeil>
</Elster>
\end{lstlisting}


%\section{Korrektheit und Vollständigkeit der Testfälle}
%Bei der Arbeit mit Testfällen ist es besonders wichtig, dass alle Angaben korrekt und vollständig sind. 
%Korrektheit bedeutet, dass die eingetragenen Werte genau den Daten aus der Quelle entsprechen und keine Fehler enthalten.
%So muss zum Beispiel die laufende Nummer korrekt übernommen werden um korrekt wiedergegeben werden zu können.
%Die Vollständigkeit bezieht sich darauf, das alle Pflichtfelder eines Testfalls ausgefüllt sind. 
%Fehlende oder unvollständige Angaben können dazu führen, dass Testfälle nicht korrekt ausgeführt werden oder im Testprozess Fehler auftreten.
%Ein weiterer wichtiger Punkt ist, dass die Testfälle klar und übersichtlich aufgebaut sind. Nur so können die Testfälle bei Bedarf problemlos kontrolliert werden.
%Eine einheitliche Struktur  hilft nicht nur den Nutzern sondern auch dem Generator, 
%da so Daten ohne schwierigkeiten verarbeitet werden können.


%\chapter{Evaluation}\label{kap:eval}

%	Fazit und Ausblick
\chapter{Fazit und Ausblick} \label{chap:Fazit und Ausblick}
Dieses Kapitel fasst die Ergebnisse der Arbeit zusammen und zeigt, welche Möglichkeiten sich für die Weiterentwicklung des Generators anbieten.
 
\section{Zusammenfassung der Arbeit}
In dieser Arbeit wurde ein Testfallgenerator für steuerliche Anwendungen entwickelt. 
Ziel war es, ein Programm zu entwickeln, das automatisch Testfälle aus vorhandenen \ac{xml} und Excel Dateien erzeugt. 
Damit sollte die Testfallerstellung schneller, einfacher und weniger fehleranfällig werden, da manuelle Eingaben oft aufwendig sind und zu Fehlern führen können.

Am Anfang der Arbeit wurden die Anforderungen an den Generator festgelegt. 
Dazu gehörte, dass die Daten aus den Dateien korrekt übernommen werden, dass die erzeugten Testfälle immer gleich aufgebaut sind und dass auch Sonderfälle wie fehlende Eingaben oder falsche Werte berücksichtigt werden. 
Die Umsetzung erfolgte mit Java. Für die Verarbeitung von \ac{xml} Dateien wurden die Werkzeuge \ac{jaxb} und \ac{dom} eingesetzt, während Excel Dateien mit der Bibliothek Apache POI gelesen wurden. 
Alle Testfälle werden in einer einheitlichen Struktur gespeichert, sodass sie später problemlos für Tests genutzt werden können.

Die Logik des Generators ist klar aufgebaut. Zuerst werden die Dateien eingelesen und überprüft. 
Danach werden daraus Testfälle erstellt. Diese werden durchnummeriert und in einem passenden Format gespeichert. 
Auf diese Weise entstehen viele Testfälle in kurzer Zeit, die für verschiedene Szenarien genutzt werden können. 
%Mithilfe von JUnit-Tests wurde geprüft, ob der Generator die Daten korrekt verarbeitet und ob die Testfälle vollständig sind. 

%Zusammenfassend lässt sich sagen, dass der Generator die gestellten Anforderungen erfüllt und die Testfallerstellung erleichtert. 
%Er sorgt dafür, dass Testfälle schnell, zuverlässig und nachvollziehbar erstellt werden können. 
Der Testfallgenerator erfüllt die wesentlichen Anforderungen und unterstützt die schnellere sowie fehlerfreie Erstellung von Testfällen.
Die automatische Verarbeitung von \ac{xml} und Excel Dateien spart Zeit und erleichtert die Arbeit.

Trotzdem gibt es auch Schwächen. Der Testfallgenerator kann keine fehlerhaften Testfälle erkennen oder gezielt erstellen. 
Das kann ein Nachteil sein, wenn man testen will, wie ein System auf falsche Eingaben reagiert. Außerdem unterstützt der Testfallgenerator nur \ac{xml} und Excel, andere Formate wie CSV oder ähnliche fehlen. 
Auch bei sehr großen Dateien könnte die Leistung besser und effizienter sein.
Die Bedienbarkeit des Testfallgeneratos ist noch ausbaufähig, da die Benutzerführung teilweise unübersichtlich ist und dadurch die Bedienung erschwert wird.

Insgesamt ist der Testfallgenerator eine gute Unterstützung, doch vor allem in der Fehlerbehandlung und der Flexibilität, gibt es noch Verbesserungsmöglichkeiten.

\section{Ideen für Weiterentwicklung}
Der Testfallgenerator erfüllt die in dieser Arbeit gesetzten Anforderungen, dennoch gibt es verschiedene Möglichkeiten, das System in Zukunft zu erweitern und an neue Anforderungen anzupassen. 
Solche Weiterentwicklungen sind sinnvoll, um den praktischen Nutzen des Generators zu erhöhen und den Einsatzbereich zu vergrößern.

Ein erster Ansatz wäre die Unterstützung weiterer Eingabeformate. Derzeit verarbeitet der Generator \ac{xml} und Excel Dateien. 
In vielen Unternehmen werden jedoch auch andere Formate wie \ac{csv} Dateien oder Datenbanken genutzt. Wenn der Generator diese Quellen ebenfalls einlesen könnte, würde er deutlich flexibler einsetzbar sein. 
Auch eine Kombination verschiedener Datenquellen wäre denkbar, um Testfälle aus unterschiedlichen Systemen zusammenzuführen.

Ein weiterer wichtiger Punkt ist die Anbindung an automatisierte Testumgebungen. 
Aktuell erzeugt der Generator die Testfälle, die anschließend manuell in Testsysteme eingebunden werden müssen. 
Würde eine direkte Schnittstelle zu gängigen Test Frameworks bestehen, könnten die erzeugten Testfälle automatisch ausgeführt werden. 
Dadurch lässt sich der gesamte Testprozess beschleunigen und effizienter gestalten.
Darüber hinaus könnte die Bedienbarkeit des Generators verbessert werden. 
Bislang läuft die Steuerung hauptsächlich über den Code. Eine grafische Benutzeroberfläche würde den Umgang deutlich vereinfachen, da auch Personen ohne tiefere Programmierkenntnisse den Generator nutzen könnten. 
Eine solche Oberfläche könnte zum Beispiel das Auswählen der Eingabedateien, das Festlegen von Parametern oder das Anzeigen der generierten Testfälle erleichtern.
Auch die Logik zur Erzeugung von Sonder und Fehlerfällen könnte weiter ausgebaut werden. 
Momentan werden grundlegende Fälle berücksichtigt, wie fehlende Eingaben oder falsche Werte. 
In Zukunft wäre es möglich, komplexere Szenarien zu erzeugen, die realistische Fehlerbilder besser abbilden. 
So könnten beispielsweise Abhängigkeiten zwischen Eingabefeldern geprüft oder seltene Spezialfälle berücksichtigt werden. Dies würde die Qualität der erzeugten Tests weiter erhöhen.
Zusätzlich ließe sich überlegen, den Generator modular zu gestalten. 
So könnten einzelne Teile wie die Verarbeitung bestimmter Datenformate oder die Logik zur Testfallerstellung leicht ausgetauscht oder erweitert werden. 
Damit wäre der Generator langfristig besser anpassbar und könnte schrittweise an neue Anforderungen angepasst werden.

Insgesamt bieten sich damit mehrere sinnvolle Ansätze für die Weiterentwicklung. Die Unterstützung weiterer Formate, die Anbindung an automatisierte Tests, eine bessere Bedienbarkeit und ein Ausbau der Testlogik. 
Mit diesen Erweiterungen könnte der Generator nicht nur breiter eingesetzt werden, sondern auch einen noch größeren Beitrag zur Qualitätssicherung leisten.

%% --- Beginn Anhang ---
%\begingroup
%\let\cleardoublepage\clearpage % verhindert leere Seite
%\clearpage
%
%\pagenumbering{Roman} % römische Seitenzahlen im Anhang
%\setcounter{page}{6}
%\appendix
%\ihead{\appendixname~\thechapter} % Kopfzeile anpassen
%
%%\chapter{Anhang}
\chapter*{Anhang}
\addcontentsline{toc}{chapter}{Anhang}


\begin{figure}[H]
    \centering
    \includegraphics[width=0.8\textwidth]{./img/Organigramm_OFD_2025.jpg}
    \caption{Oberfinanzdirektion Baden-Württemberg Organigramm}
    \label{fig:bild1}
\end{figure}
%
%\endgroup
%% --- Ende Anhang ---

% --- Beginn Anhang ---
%\clearpage
%\KOMAoptions{open=any} % stellt sicher, dass kein Kapitel gezwungen rechts startet
%\appendix
%
%% Kopfzeile für Anhang
%\ihead{\appendixname~\thechapter}
%\ohead{\headmark}
%
%% Falls KOMA oder fancyhdr noch eine leere Seite erzeugt:
%\makeatletter
%\renewcommand*{\chapter}{%
%  \if@openright\ifodd\value{page}\else\hbox{}\newpage\fi\fi
%  \thispagestyle{headings}%
%  \global\@topnum\z@
%  \@afterindentfalse
%  \secdef\@chapter\@schapter}
%\makeatother
%
%% römische Nummerierung im Anhang
%\pagenumbering{Roman}
%
%% Jetzt den Anhang einbinden
%%\chapter{Anhang}
\chapter*{Anhang}
\addcontentsline{toc}{chapter}{Anhang}


\begin{figure}[H]
    \centering
    \includegraphics[width=0.8\textwidth]{./img/Organigramm_OFD_2025.jpg}
    \caption{Oberfinanzdirektion Baden-Württemberg Organigramm}
    \label{fig:bild1}
\end{figure}
% --- Ende Anhang ---


\clearpage

% Der Anhang beginnt hier - jedes Kapitel wird alphabetisch aufgezählt. (Anhang A, B usw.)
\pagenumbering{Roman} % Arabische Seitenzahlen
\setcounter{page}{6}
\appendix
\ihead{\appendixname~\thechapter} % Neue Header-Definition

% appendix.tex einziehen
%\chapter{Anhang}
\chapter*{Anhang}
\addcontentsline{toc}{chapter}{Anhang}


\begin{figure}[H]
    \centering
    \includegraphics[width=0.8\textwidth]{./img/Organigramm_OFD_2025.jpg}
    \caption{Oberfinanzdirektion Baden-Württemberg Organigramm}
    \label{fig:bild1}
\end{figure}
\ihead{\appendixname~\thechapter} % Neue Header-Definition

%Beigabenverzeichnis
%\ihead{Beigaben}
%\input{beigaben}






%Literaturverzeichnis
\ihead{Literaturverzeichnis}
%\printbibliography[title=Literaturverzeichnis] %Standard aus Vorlage
\printbibliography[title=Literaturverzeichnis]
%\printbibliography[heading=bibintoc, type=book, title={Literaturverzeichnis}]
%\printbibliography[heading=bibintoc, type=online, title={Elektronische Quellen}]
%\printbibliography[heading=bibintoc, type=article, title={Artikel}]
%\printbibliography[heading=bibintoc, type=thesis, title={Theses}]


\cleardoublepage

\end{document}