% !TEX root =  master.tex

%		LANGUAGE SETTINGS AND FONT ENCODING 
%
\usepackage[ngerman]{babel} 	% German language
\usepackage[utf8]{inputenc}
\usepackage[german=quotes]{csquotes} 	% correct quotes using \enquote{}
\usepackage[T1]{fontenc}

%\usepackage[english]{babel}   % For english language
%\usepackage{csquotes} 	% Richtiges Setzen der Anführungszeichen mit \enquote{}

% Zwei eigene Befehle zum Setzen von Autor und Titel. Ausserdem werden die PDF-Informationen richtig gesetzt.
\newcommand{\TitelDerArbeit}[1]{\def\DerTitelDerArbeit{#1}\hypersetup{pdftitle={#1}}}
\newcommand{\AutorDerArbeit}[1]{\def\DerAutorDerArbeit{#1}\hypersetup{pdfauthor={#1}}}
\newcommand{\Firma}[1]{\def\DerNameDerFirma{#1}}
\newcommand{\Kurs}[1]{\def\DieKursbezeichnung{#1}}

% Correct superscripts 
\usepackage{fnpct}

%		CALCULATIONS
%
\usepackage{calc} % Used for extra space below footsepline

%		BIBLIOGRAPHY SETTINGS
%
% Uncomment the next three lines for author-year-style with footnotes (Chicago)
%\usepackage[backend=biber, autocite=footnote, style=authortitle, dashed=false]{biblatex}%vorher authoryear
%Use Author-Year-Cites with footnotes
%\AdaptNoteOpt\footcite\multfootcite   %will add  separators if footcite is called multiple consecutive times 
%\AdaptNoteOpt\autocite\multautocite % will add  separators if autocite is called multiple consecutive times

% Uncomment the next line for IEEE-style 
% \usepackage[backend=biber, autocite=inline, style=ieee]{biblatex} 	% Use IEEE-Style (e.g. [1])

% Uncomment the next line for alphabetic style 
% \usepackage[backend=biber, autocite=inline, style=alphabetic]{biblatex} 	% Use alphabetic style (e.g. [TGK12])

% Uncomment the next two lines vor Harvard-Style 
\usepackage[natbib=true, backend=biber, style=authoryear, dashed=false]{biblatex} 	
%\DeclareLanguageMapping{german}{german-apa}


\DefineBibliographyStrings{ngerman}{  %Change u.a. to et al. (german only!)
	andothers = {{et\,al\adddot}},
}

\renewcommand{\bibleftparen}{[}
\renewcommand{\bibrightparen}{]}

\let\cite\parencite

%%%für die Umbrüche der URLs im Literaturverzeichnis
\apptocmd{\UrlBreaks}{\do\f\do\m}{}{}
\setcounter{biburllcpenalty}{9000}% Kleinbuchstaben
\setcounter{biburlucpenalty}{9000}% Großbuchstaben

\setlength{\bibparsep}{\parskip}	%add some space between biblatex entries in the bibliography
\addbibresource{bibliography.bib}	%Add file bibliography.bib as biblatex resource

%		FOOTNOTES 
%
% Count footnotes over chapters
\usepackage{chngcntr}
\counterwithout{footnote}{chapter}

\usepackage{tablefootnote} %damit Fußnoten in Tabellen richtig angezeigt werden

%	ACRONYMS
%%%
%%% WICHTIG: Installieren Sie das neueste Acronyms-Paket!!!
%%%
\makeatletter
\usepackage[printonlyused]{acronym}
\@ifpackagelater{acronym}{2015/03/20}
  {%
    \renewcommand*{\aclabelfont}[1]{\textbf{\textsf{\acsfont{#1}}}}
  }%
  {%
  }%
\makeatother

%		LISTINGS
\usepackage{listings}	%Format Listings properly
\usepackage{color} %für das Styling in anderen Farben
\usepackage{xcolor}

\renewcommand{\lstlistingname}{Quelltext}
\renewcommand{\lstlistlistingname}{Quelltextverzeichnis}

\lstset{numbers=left,
  numberstyle=\tiny,
  captionpos=t,
  basicstyle=\ttfamily\small,
  escapeinside={(*@}{@*)},
  literate= %für die Umlaute in Listings
  {Ö}{{\"O}}1
  {Ä}{{\"A}}1
  {Ü}{{\"U}}1
  {ß}{{\ss}}1
  {ö}{{\"o}}1
  {ü}{{\"u}}1
  {ä}{{\"a}}1
  {~}{{\textasciitilde}}1
}
\definecolor{codeblue}{rgb}{0.25,0.5,0.75}
\definecolor{codered}{rgb}{0.8,0,0}
\definecolor{codegreen}{rgb}{0,0.6,0} 
\lstdefinestyle{stylePython}{
    backgroundcolor=\color{white},       
    belowcaptionskip=1\baselineskip,
    breaklines=true,
    frame=L,
    xleftmargin=\parindent,
    language=Python,
    showstringspaces=false,
    basicstyle=\footnotesize\ttfamily,
    commentstyle=\color{codegreen},
    keywordstyle=\color{blue},
    identifierstyle=\color{black},
    stringstyle=\color{red},
    numberstyle=\tiny\color{blue},
    morekeywords={with, as}
}
\lstdefinestyle{styleLog}{
    backgroundcolor=\color{white},       
    belowcaptionskip=1\baselineskip,
    breaklines=true,
    frame=L,
    xleftmargin=\parindent,
    showstringspaces=false,
    basicstyle=\small
}
\lstdefinestyle{bashstyle}{
    language=bash,
    basicstyle=\ttfamily\footnotesize,
    keywordstyle=\color{blue}\bfseries,
    commentstyle=\color{gray}\itshape,
    stringstyle=\color{orange},
    backgroundcolor=\color{gray!10},
    frame=single,
    rulecolor=\color{black},
    breaklines=true,
    postbreak=\mbox{\textcolor{red}{$\hookrightarrow$}\space},
    showstringspaces=false,
    tabsize=2,
    captionpos=b,
}
\lstdefinestyle{styleSQL}{
    backgroundcolor=\color{white},
    belowcaptionskip=1\baselineskip,
    breaklines=true,
    frame=L,
    xleftmargin=\parindent,
    language=SQL,
    showstringspaces=false,
    basicstyle=\footnotesize\ttfamily,
    commentstyle=\color{codegreen},
    keywordstyle=\color{codeblue}\bfseries,
    identifierstyle=\color{black},
    stringstyle=\color{codered},
    numberstyle=\tiny\color{gray},
    morekeywords={IF, EXISTS, BEGIN, END, VARCHAR, INT, PRIMARY, KEY, FOREIGN, REFERENCES, CREATE, TABLE, INSERT, INTO, VALUES, SELECT, FROM, WHERE, JOIN, ON, GROUP, BY, ORDER, ASC, DESC, UPDATE, SET, DELETE},
}

\lstdefinelanguage{JavaScript}{
    morekeywords={typeof, new, true, false, catch, function, return, null, catch, switch, var, if, in, while, do, else, case, break},
    morecomment=[l]//,
    morecomment=[s]{/*}{*/},
    morestring=[b]",
    morestring=[b]',
}

\lstdefinelanguage{JSX}{
    language=JavaScript,
    morekeywords={
        className, render, return, props, state, useState, useEffect, useContext, useRef, useMemo, useCallback,
        React, Component, Fragment, setState, export, default, import, from, const, let, Grid, Box, SearchForm, DataGrid
    },
    sensitive=true,
    morecomment=[s]{/*}{*/},
    morestring=[b]',
    morestring=[b]"
    % Keine moredelim-Zeilen mehr
}
\lstdefinestyle{styleJavaSpring}{
    backgroundcolor=\color{white},       
    belowcaptionskip=1\baselineskip,
    breaklines=true,
    frame=L,
    xleftmargin=\parindent,
    language=Java,
    showstringspaces=false,
    basicstyle=\footnotesize\ttfamily,
    commentstyle=\color{codegreen},
    keywordstyle=\color{blue},
    identifierstyle=\color{black},
    stringstyle=\color{red},
    numberstyle=\tiny\color{blue},
    morekeywords={
        @Autowired, @Service, @Component, @Controller, @RestController, @Repository, @RequestMapping,
        @GetMapping, @PostMapping, @PutMapping, @DeleteMapping,
        SpringApplication, SpringBootApplication,
        ResponseEntity, RequestBody, PathVariable, RequestParam,
        List, Map, Optional
    }
}

\lstdefinestyle{styleReact}{
    backgroundcolor=\color{white},
    belowcaptionskip=1\baselineskip,
    breaklines=true,
    frame=L,
    xleftmargin=\parindent,
    language=JSX,
    showstringspaces=false,
    basicstyle=\footnotesize\ttfamily,
    commentstyle=\color{codegreen},
    keywordstyle=\color{blue},
    identifierstyle=\color{black},
    stringstyle=\color{red},
    numberstyle=\tiny\color{blue},
}

%		EXTRA PACKAGES
\usepackage{graphicx} % use various graphics formats
\usepackage[german]{varioref} 	% nicer references \vref
\usepackage{caption}	%better Captions
\usepackage{booktabs} %nicer Tabs
\usepackage{array}
%\newcolumntype{P}[1]{>{\raggedright\arraybackslash}p{#1}}
\usepackage{setspace} %für Zeilenabstand
\usepackage{hyphenat}
\usepackage{microtype}
\usepackage{amsmath}

% Tabelle
\usepackage{longtable}

%		ALGORITHMS
\usepackage{algorithm}
\usepackage{algpseudocode}
\renewcommand{\listalgorithmname}{Algorithmenverzeichnis }
\floatname{algorithm}{Algorithmus}

% Pages in landscape format
\usepackage{pdflscape}

%		FONT SELECTION: Entweder Latin Modern oder Times / Helvetica
%\usepackage{lmodern} %Latin modern font
\usepackage{mathptmx}  %Helvetica / Times New Roman fonts (2 lines)
\usepackage[scaled=.92]{helvet} %Helvetica / Times New Roman fonts (2 lines)

%PAGE HEADER / FOOTER
%Warning: There are some redefinitions throughout the master.tex-file!  DON'T CHANGE THESE REDEFINITIONS!
\RequirePackage[automark,headsepline,footsepline]{scrlayer-scrpage}
%\pagestyle{scrlayer-scrpage}
\renewcommand*{\pnumfont}{\upshape\sffamily}
\renewcommand*{\headfont}{\upshape\sffamily}
\renewcommand*{\footfont}{\upshape\sffamily}
\renewcommand{\chaptermarkformat}{}
\RedeclareSectionCommand[beforeskip=0pt]{chapter}
\clearscrheadfoot

\ifoot[\rule{0pt}{\ht\strutbox+\dp\strutbox}DHBW Karlsruhe]{\rule{0pt}{\ht\strutbox+\dp\strutbox}DHBW Karlsruhe}
\ofoot[\rule{0pt}{\ht\strutbox+\dp\strutbox}\pagemark]{\rule{0pt}{\ht\strutbox+\dp\strutbox}\pagemark}

\ohead{\headmark}

% 		HYPERREF
%
\usepackage[
	hidelinks=true % keine roten Markierungen bei Links
]{hyperref} %am Ende laden, ansonsten kommt es zu Fehlern

\makeatletter %damit Weiterleitung ins Glossar auf richtige Zeile erfolgt
 \newcommand{\linkdest}[1]{\Hy@raisedlink{\hypertarget{#1}{}}}
\makeatother

\setcounter{secnumdepth}{2}

%Behebung des Hook Errors beim Einbinden von lstlistings
\makeatletter %https://groups.google.com/g/comp.text.tex/c/nMUAQcY0N30
\ExplSyntaxOn
\cs_set_protected:Npn \para_end: {
\scan_stop:
\mode_if_horizontal:TF {
\mode_if_inner:F {
\tex_unskip:D
\hook_use:n{para/end}
\@kernel@after@para@end
\mode_if_horizontal:TF {
\if_int_compare:w 0 < \tex_lastnodetype:D
\tex_kern:D \c_zero_dim
\fi:
\tex_par:D
\hook_use:n{para/after}
\@kernel@after@para@after
}
{ \msg_error:nnnn { hooks }{ para-mode }{end}{horizontal} }
}
}
\tex_par:D
}
\cs_set_eq:NN \par \para_end:
\cs_set_eq:NN \@@par \para_end:
\cs_set_eq:NN \endgraf \para_end:
\ExplSyntaxOff
\makeatother

% adjust space before \paragraph{}
% https://tex.stackexchange.com/questions/4891/how-do-i-control-the-spacing-above-a-new-paragraph
\makeatletter
\renewcommand{\paragraph}{%
  \@startsection{paragraph}{4}%
  {\z@}{1.5ex \@plus 1ex \@minus .2ex}{-1em}%
  {\normalfont\normalsize\bfseries}%
}
\makeatother

% add "Quelltext" to the listings
% https://tex.stackexchange.com/questions/247038/change-lstlistoflistings-numbering
\makeatletter
\renewcommand{\l@lstlisting}[2]{%
  \@dottedtocline{1}{0em}{1.5em}{\lstlistingname\ #1}{#2}%
}
\makeatother

% add chapter to algorithm numbering
% https://tex.stackexchange.com/questions/124902/algorithm-with-chapter-number
\makeatletter 
\renewcommand\thealgorithm{\thechapter.\arabic{algorithm}} 
\@addtoreset{algorithm}{chapter} 
\makeatother

