\lstdefinelanguage{XML}{
  morestring=[b]",
  morecomment=[s]{<!--}{-->},
  stringstyle=\color{red},
  identifierstyle=\color{blue},
  morekeywords={xmlns,version,type}
}

\lstset{
  language=XML,
  basicstyle=\ttfamily\tiny,    % kleine Schrift
  numbers=left,
  numberstyle=\tiny,
  keywordstyle=\color{blue},
  commentstyle=\color{gray},
  stringstyle=\color{red},
  breaklines=true,
  frame=single,
  captionpos=b
}

\chapter{Tests} \label{chap:Tests}
In diesem Kapitel werden die Funktionstests des Generators mithilfe von JUnit vorgestellt. 
Dabei wird insbesondere auf die beispielhaften Testfallausgaben sowie auf die Korrektheit und Vollständigkeit der generierten Testfälle eingegangen.

\section{Funktionstests des Generators (JUnit Tests)}
Um die Funktionsweise des Generators zu überprüfen, wurden JUnit-Tests eingesetzt.
Mit diesen Tests kann kontrolliert werden ob die Eingabedaten aus \ac{xml} und Excel Dateien korrekt eingelesen und in die vorgesehene interne Struktur übertragen werden. 
Zusätzlich wurde getestet, 
ob die erzeugten Testfälle dem festgelegten Schema entsprechen und die Ausgaben vollständig und fehlerfrei sind. 
Die Tests berücksichtigen verschiedene Szenarien, wie beispielsweise das Einlesen unterschiedlicher \ac{xml} Elemente mit variierenden Attributen sowie Excel Dateien mit diversen Zellinhalten. 
Die Testfälle sind so aufgebaut, dass sie die Vollständigkeit und Korrektheit der eingelesenen Daten sicherstellen. 
Hierbei wird geprüft, ob alle Werte korrekt den jeweiligen internen Objekten zugeordnet werden. 
Die automatisierte Ausführung der Tests ermöglicht eine schnelle und wiederholbare Überprüfung der Funktionsfähigkeit des Generators, sodass nach Änderungen am Code zeitnah die Zuverlässigkeit des Systems bestätigt werden kann.

\section{Beispielhafte Testfallausgaben} \label{name: Beispielhafte Testfallausgaben}
Um die Arbeitsweise des Generators besser verstehen zu können, werden im Folgenden einige Beispiele der erzeugten Testfälle gezeigt. 
Diese Beispiele machen deutlich, wie die Testfälle aufgebaut sind und welche Informationen sie enthalten. 
Jeder Testfall hat dabei eine eindeutige laufende Nummer, eine kurze Beschreibung und die dazugehörigen Eingabedaten.
Die eingelesenen Informationen sind im DatenTeil zu finden, GW1, GW2 und GW4 stehen für die einzelnen Formulare innerhalb der Erklärung. 
Die Feldkennnummer wird mit 'nr' abgekürzt und Felder ohne Kennnummer bekommen ihre Bezeichung als Feldkennnummer.

Zur Veranschaulichung beispielhafter Testfallausgaben wurde \ac{xml} verwendet, um den Aufbau und die Struktur darzustellen. \newline

\begin{lstlisting}[caption={Beispielhafte Testfallausgaben in XML}] 
<Elster xmlns="http://www.elster.de/elsterxml/schema/v11">
<Verfahren>ElsterErklaerung</Verfahren>
<DatenArt>GrundSteuerBW</DatenArt>
<DatenTeil>
<Erklaerung LfdNr="4">
<GW1>
<Feld nr="lfdNr" wert="4"/>
<Feld nr="Bezeichnung" wert="Sachbearbeitung"/>
<Feld nr="Kurzbeschreibung" wert="Bauerwartungsland"/>
<Feld nr="Stichtag" wert="Sat Jan 01 00:00:00 CET 2022"/>
<Feld nr="Aktenzeichen" wert="083500120680040016"/>
<Feld nr="Finanzamt" wert="Karlsruhe-Stadt"/>
<Feld nr="E7401124" wert="Adalbert-Stifter-Straße"/>
<Feld nr="E7401125" wert="9"/>
<Feld nr="E7401131" wert="Wohnungseigentum 4"/>
<Feld nr="E7401121" wert="76199"/>
<Feld nr="E7401122" wert="Karlsruhe"/>
<Feld nr="E7401141" wert="Gaggenau"/>
<Feld nr="E7411702" wert="Bei Rückfragen bitte telefonisch am Vormittag 0761123456"/>
<Feld nr="lfdNr2" wert="001"/>
<Feld nr="E7404518" wert="14.03.1967"/>
<Feld nr="E7404513" wert="Tom"/>
<Feld nr="E7404511" wert="Maus"/>
<Feld nr="E7404524" wert="Im Efeu"/>
<Feld nr="E7404571" wert="2"/>
<Feld nr="lfdNr3" wert="002"/>
<Feld nr="E74045182" wert="15.04.1967"/>
<Feld nr="E74045132" wert="Jerry"/>
<Feld nr="E74045112" wert="Maus "/>
<Feld nr="E74045242" wert="Im Näpfle"/>
<Feld nr="E74045252" wert="5"/>
<Feld nr="E74045402" wert="74078"/>
<Feld nr="E74045222" wert="Heilbronn"/>
</GW1>
<GW2>
<Feld nr="E7421322" wert="1"/>
<Feld nr="E7403010" wert="71"/>
<Feld nr="E7403011" wert="730"/>
</GW2>
<GW4/>
</Erklaerung>
</DatenTeil>
</Elster>
\end{lstlisting}


%\section{Korrektheit und Vollständigkeit der Testfälle}
%Bei der Arbeit mit Testfällen ist es besonders wichtig, dass alle Angaben korrekt und vollständig sind. 
%Korrektheit bedeutet, dass die eingetragenen Werte genau den Daten aus der Quelle entsprechen und keine Fehler enthalten.
%So muss zum Beispiel die laufende Nummer korrekt übernommen werden um korrekt wiedergegeben werden zu können.
%Die Vollständigkeit bezieht sich darauf, das alle Pflichtfelder eines Testfalls ausgefüllt sind. 
%Fehlende oder unvollständige Angaben können dazu führen, dass Testfälle nicht korrekt ausgeführt werden oder im Testprozess Fehler auftreten.
%Ein weiterer wichtiger Punkt ist, dass die Testfälle klar und übersichtlich aufgebaut sind. Nur so können die Testfälle bei Bedarf problemlos kontrolliert werden.
%Eine einheitliche Struktur  hilft nicht nur den Nutzern sondern auch dem Generator, 
%da so Daten ohne schwierigkeiten verarbeitet werden können.

